\documentclass{book}

\begin{document}

CROWS AND OWLS
CROWS AND OWLS
Here, then, begins Book III, called "Crows and
Owls," which treats of peace, war, and so forth. The
first verse runs:
Reconciled although he be,
Never trust an enemy.
For the cave of owls was burned,
When the crows with fire returned.
"How was that?" asked the princes, and Vish-
nusharman told the following story.
In the southern country is a city called Earth-
Base. Near it stands a great banyan tree with count-
less branches. And in the tree dwelt a crow-king
named Cloudy with a countless retinue of crows.
There he made his habitation and spent his time.
Now a rival king, a great owl named Foe-Crusher,
had his fortress and his habitation in a mountain
cave, and he had an unnumbered retinue of owls.
This owl-king cherished a grudge, so that whenever
he met a crow in his airings, he killed him and passed
on. In this way his constant aggression gradually
spread rings of dead crows about the banyan tree.
Nor is this surprising. For the proverb says:
291
If you permit disease or foe
To march unheeded, you may know
That death awaits you, sure if slow.
Now one day Cloudy summoned all his counselors
and said: "Gentlemen, as you are aware, our enemy
is arrogant, energetic, and a judge of occasions. He
always comes at nightfall to work havoc in our ranks.
How, then, can we counter-attack? For we do not see
at night, and in the daytime we cannot discover his
fortress. Otherwise, we might go there and strike a
blow. What course, then, shall we adopt? There are
six possibilities---peace, war, change of base, en-
trenchment, alliances, and duplicity."
And they replied: "Your Majesty does well to
put this question. For the saying goes:
Good counselors should tell their king,
Unasked, a profitable thing;
If asked, they should advise.
While flatterers who shun the true
(Which in the end is wholesome, too)
Are foemen in disguise.
Therefore it is now proper to confer in secret session/'
Then Cloudy started to consult severally his five
ancestral counselors, whose names were Live-Again,
Live-Well, Live-Along, Live-On, and Live-Long. And
first of all he questioned Live-Again: "My worthy
sir, what is your opinion under the circumstances?"
And Live-Again replied: "O King, one should not
make war with a powerful enemy. And this one is
powerful and knows when to strike. Therefore make
peace with him. For the saying goes:
Bow your head before the great,
Lifting it when times beseem,
And prosperity will flow
Ever onward, like a stream.
And again:
Make your peace with powerful foes
Who are rich and good and wise,
Who are seasoned conquerors,
In whose home no discords rise.
Make your peace with wicked men,
If your life endangered be;
Life, itself first made secure,
Gives the realm security.
And again:
Make your peace with him whose wont
'Tis to conquer in a fight;
Other foes will bend their necks
To you, fearful of his might.
Even with equals make your peace;
Victory is often given
Whimsically; take no risks
Says the current saw in heaven.
Even with equals victory
Whimsically may alight.
Try three other methods first;
Only in extremis fight.
And yet again:
See! The bully to whose soul
Power is all, and peace is not,
Clashing with an equal foe,
Crumbles like an earthen pot.
Land and friends and gold at most
Have been won when battles cease;
If but one of these should fail,
It is best to live in peace.
When a lion digs for moles
Hiding in their pebbly house,
He is apt to break his nails,
And at best he gets a mouse.
Therefore, where no prize is won
And a healthy fight is sure,
Never stir a quarrel, but
Whatsoe'er the cost, endure.
By a stronger foe assailed,
Bend as bends the river reed;
Do not strike, as serpents do,
If you wish your luck to speed.
Imitators of the reed
Slowly win to glory's peak;
But the luckless serpent-men
Only earn the death they seek.
Shrink like turtles in their shells,
Taking blows if need there be;
Raise your head from time to time
Like the black snake, warily.
To sum it up:
Never struggle with the strong
(If you wish to know my mind)
Who has ever seen a cloud
Baffle the opposing wind ?"
Having heard this view, the king said to Live-
Well: "My worthy sir, I desire to hear your opinion
also." And Live-Well said: "O King, I disagree.
Inasmuch as the enemy is cruel, greedy, and unprinci-
pled, you should most certainly not make peace with
him. For the proverb says:
With foes unprincipled and false
'Tis vain to seek accommodation:
Agreements bind them not; and soon
They show a wicked transformation.
Therefore you should, in my judgment, fight with
him. You know the saying:
'Tis easy to uproot a foe
Contemning fighters, never steady,
Cruel and greedy, slothful, false,
Foolish and fearful and unready.
"But more than this---we have been humiliated by
him. Therefore, if you propose peace, he will be angry
and will employ violence again. There is a saying:
The truculence of fevered foes
By gentle measures is abetted:
What wise physician tries a douche?
He knows that fever should be sweated.
Conciliation simply makes
A foeman's indignation splutter,
Like drops of water sprinkled on
A briskly boiling pan of butter.
Besides,  the  previous  speaker's  point  about  the
strength of the enemy is not decisive.
The smaller often slays the great
By showing energy and vigor:
The lion kills the elephant,
And rules with unrestricted rigor.
And more than that:
Foes indestructible by might
Are slain through some deceptive gesture,
As Bhima strangled Kichaka,
Approaching him in woman's vesture.
And yet again:
When kings are merciless as death,
All foes are quick to knuckle under;
Quick, too, to kill the kings who fall
Into compassion's fatal blunder.
And he whose sun of glory sets
Before the glory of another
Is born in vain; he wastes for naught
The youthful vigor of his mother.
For Regal Splendor, unbesmeared
With foemen's blood as rich cosmetic,
Though dear, is insufficient for
Ambitions truly energetic.
And in a kingdom unbedewed
With foemen's blood in slaughter gory,
And hostile women's falling tears,
The king enjoys no living glory."
Having heard this view, the king put the question
to Live-Along: "My worthy sir, pray express your
opinion also." And Live-Along said: "O King, the
enemy is vicious and powerful and unscrupulous.
Therefore you should make neither peace nor war
with him. Only a change of base can be recommend-
ed. For the saying goes:
With vicious foemen, proud of power,
From hindering scruples free,
Adopt a change of base, not peace
Nor war, for victory.
Now change of base is known to be
No single thing, but twin
Retreat, to save imperiled life;
Invasion, planned to win.
A warlike and ambitious king
May choose 'twixt April and
November---other months are barred
To invade the hostile land.
For storming-parties---so the books
Prescribe---all times are fair,
If hostile forces show distress,
And lay some weakness bare.
A king should put his realm in charge
Of heroes strong and fit;
Then pounce upon the hostile land,
When spies have peopled it.
The case in hand requires, O King,
The base-change called Retreat,
Not peace nor war; the foe is vile,
And very hard to beat.
"Furthermore, a recessive movement is made,
says the science of ethics, with due regard to cause
and effect. The point is thus expressed in poetry:
When rams draw back, their butting fiercer stings;
The crouching king of beasts more deadly springs:
So wise dissemblers, holding vengeance sure,
In dumb communion with their hearts, endure.
And once again:
A king, abandoning his realm
To foes of fighting worth,
Preserves his life, as Fight-Firm did,
And later rules the earth.
And so, to sum it up:
The weak who, struggling with the strong,
Are not too proud to fight,
Bring great rejoicing to their foes,
And on their kinsmen, blight.
"Therefore, since you are engaged with a powerful
foe, there is occasion for a change of base. It is no
time for peace or war/'
When he had listened to this view, the king said
to Live-On: "My worthy sir, pray express your opin-
ion also." And Live-On said: "O King, I disapprove
of peace, war, and change of base, all three of them;
and particularly change of base. For
A crocodile at home
Can beat an elephant;
But if he goes abroad,
A dog can make him pant.
And again:
When stronger foes attack,
Close in your fortress stay;
But sally to relieve
Your friends, and save the day.
If, panic-struck, you flee
When foes are at the door,
And leave the land to them,
You ne'er will see it more.
One man, entrenched, can hold
A hundred foes at bay
(Strong foes at that), therefore
In your entrenchment stay.
Therefore provide your fort
With shaft and gun; adorn
It well with moat and wall,
And store abundant corn.
Stand ever firm within,
Resolved to do or die:
So, living, earn renown;
Or dead, the starry sky.
And there is a further consideration:
The union of the weak
A powerful bully stumps:
The hostile blizzard spares
The shrubs that grow in clumps.
And single trees, though huge
And posted for defense,
May be uprooted by
The stout wind's violence.
While groves of trees, where each
Receives and gives defense,
Unitedly defy
The wind's fierce violence.
Just so, one man alone,
However brave he be,
Is scorned by foes, who soon
Proceed to injury."
Having listened to this view likewise, the king
said to Live-Long: "My worthy sir, pray express
your opinion also." And Live-Long said: "O King,
from among the six possibilities, I recommend alli-
ance. Pray adopt that. For the saying goes:
Though deft and brilliant, what good end
Can you attain without a friend?
The fire that seems immortal will
Die when the fanning wind is still.
"Therefore you should stay at home and seek some
competent ally, to make a counterweight against the
enemy. But if you leave home and travel, no one will
give you so much as a friendly word. For the proverb
says:
The wind is friend to forest-fire
And causes it to flame the higher;
The same wind blows a candle out.
Who cares what poor folk are about?
"Nor is it even essential that the ally be powerful;
the alliance even of feeble folk makes for defense. You
know the saying:
However weak, a bamboo stem
From others takes, and gives to them
Strength to resist uprooting: so
Weak kings unite against a foe.
"And how much more so, if you have alliance with
the truly great! For the poet says:
Who is there whom a friendly state
With great folk does not elevate?
The raindrop, hiding in a curl
Of lotus-petal, shines like pearl.
"Thus, O King, there is no counterweight to your
enemy save in alliance. Therefore let an alliance be
concluded. Such is my opinion."
After these opinions had been given, Cloudy
bowed low before an ancient, farsighted counselor of
his race. This was a crow who had persevered to the
last page of every textbook of social ethics, and his
name was Live-Strong. "Father," said the king, "I
had a secret purpose in questioning the others in your
very presence; namely, that you might listen to every-
thing, and instruct me as to what is fitting. Pray in-
struct me in the appropriate course of action."
And Live-Strong said: "My son, all that these
have proposed is drawn from the textbooks of social
ethics, and all is highly proper, each course in its own
good time. But the present hour demands duplicity.
You have heard the saying:
You must regard with like distrust
Both peace and warlike measures; must
Seek through duplicity your goal,
With powerful foes of evil soul.
"In this way those who themselves trust nobody
and have a single eye to self-interest can win the
trust of an enemy and easily destroy him. For the
saying goes:
Shrewd enemies will cause a foe
Whom they would ruin, first to grow:
The flow of mucus by molasses
Is first increased, but later passes.
And again:
To foe, to false friend, to female
(Particularly her for sale)
The man so simple as to give
Straightforward conduct, does not live.
Proceed in pure straightforwardness
With Brahmans, with the gods no less,
With teachers, with yourself; but treat
All other creatures to deceit.
A hermit mastering his soul
May see life simple, see it whole;
Not those who thirst for carnal things,
Nor, most particularly, kings.
And so:
Strong through duplicity, you will
Preserve your habitation still;
For death will prove a friend in need,
To crush a foe possessed by greed.
"Furthermore, if a vulnerable point appears in him,
you will destroy him by being aware of it."
But Cloudy said: "Father, I do not know his
residence. So how shall I become aware of a vulner-
able point?"
And Live-Strong replied: "My son, through spies
I will reveal not only his dwelling, but also his vulner-
able point. For
Cows see a thing by sense of smell;
While Scripture serves the Brahman well;
The king perceives by means of spies:
And other creatures use their eyes.
And in this connection there is another saying:
The king, well served by spies, who knows
The functionaries of his foes,
Who knows his retinue no less,
Is never plunged in deep distress."
Then Cloudy said: "Father, what are these func-
tionaries? What is their number? And of what char-
acter are secret-service men? Pray tell me all."
And Live-Strong replied: "On these points the
sage Narada gave the following information when
questioned by King Fight-Firm. In the hostile camp
are eighteen functionaries; in one's own, fifteen.
Their conduct is discovered by assigning to each
three secret-service men, by whose efforts both friends
and enemies are kept in good control. The facts are
put in a bit of doggerel:
The foe has eighteen functionaries;
And you have five and ten:
Give each, as unknown secretaries,
Three secret-service men.
"The term 'functionary' implies a delegated task.
If this be shamefully performed, it ruins the king; if
admirably, it brings him high success.
"Now for details. The functionaries in the hostile
camp are---the counselor, the chaplain, the com-
mander-in-chief, the crown prince, the concierge, the
superintendent of the gyneceum, the adviser, the
tax-collector, the introducer, the master of cere-
monies, the director of the stables, the treasurer, the
minister for elephants, the assessor, the war-minister,
the minister for fortifications, the favorite, the for-
ester, and so forth. By sowing intrigue among these
the enemy is subdued. In one's own camp the func-
tionaries are---the queen, the queen-mother, the
chamberlain, the florist, the lord of the bedchamber,
the chief of the secret service, the star-gazer, the
court physician, the purveyor of water, the purveyor
of spices, the professor, the life-guard, the quarter-
master, the bearer of the royal umbrella, and the
geisha. It is by way of these that ruin befalls one's
own party. As the saying goes:
Professor, star-scout, and physician
Find flaws within your home position:
The madman and snake-charmer know
Points vulnerable in the foe."
"Father," said Cloudy, "what is the origin of the
deadly feud between crows and owls?"
And Live-Strong answered:  "Listen.   I will tell
you
HOW THE BIRDS PICKED A KING
Once upon a time the bird-clans gathered for
consultation. There were swans and cranes and
nightingales; there were peacocks, plovers, and owls;
there were doves and pigeons and partridges; there
were bluejays, vultures, skylarks; there were demoi-
selles and cuckoos and woodpeckers and many others.
And they said: "We have in Garuda a king, to be
sure. But he is ever intent on serving holy Vishnu,
and pays no heed to us. What is the good of a sham
king? He does not defend us when we are in genuine
distress---when we are caught in traps, for instance.
There is a saying:
Only one, but anyone
Is my king, when all is done
Only one who will restore
Health and joy I felt before:
Anyone, but only one
For the moon a single sun.
"Any other is king only in name. As the poet says:
Let him calm the panting breath
Of his people, quivering
Under blows; or he is Death
Masquerading as a king.
And again:
These six should every man avoid
Like leaky ships at sea
A dull professor; and a priest
Without theology;
A king who does not give defense;
A wife whose tongue can slash;
A cowboy hankering for town;
A barber after cash.
We must therefore pitch upon someone else as king
of the birds."
Thereupon, observing that the owl had a vener-
able appearance, they all said: "Let this owl be our
king. And let a plentiful supply be provided of all
substances prescribed for the anointing of a king."
Straightway water was brought from various holy
streams; a bouquet of one hundred and eight roots
was provided, including the one marked with a
wheel and the yellow-stemmed lotus; and the lion-
throne was set in place. Moreover, there was drawn
on the ground a relief map of the seven continents,
oceans, and mountains. A tiger-skin was spread.
Golden jars were filled with five twigs, blossoms and
grains; oblations were prepared; the most eminent
bards chanted poetry. Furthermore, Brahmans,
skilled in reciting the four Vedas, also chanted, while
maidens sang songs, sweet holiday songs being their
specialty. In the forefront was prepared a vessel of
consecrated rice set off with white mustard, parched
grain, rice-grains, yellow pigment, wreaths of flowers,
conch-shells, and so forth. The materials for lustra-
tion ceremonies were provided, and holiday drums
rumbled. In the midst of a consecrated spot strewn
with potash stood the lion-throne, adorned by the
person of the owl as he waited the anointing.
At that moment a crow came into the assembly
from nobody knew where, announcing his entrance
with a raucous caw. And he thought: "Well, well!
What means this gathering of all the birds, and this
great festival ?"
But when the birds saw him, they whispered to-
gether: "He is the shrewdest of the birds, they say.
So let us have a speech from him, too. For the prov-
erb says:
Of men, the barber smartest is;
The jackal, of the beasts;
The crow is cleverest of birds;
The White-Robe, of the priests.
And besides:
Concerted counsels of the wise,
If heedfully thought through,
Will never founder, being sound
From every point of view."
So the birds said to the crow: "You know, the
birds have no king. They have therefore decided
unanimously to anoint this owl as their supreme
monarch. Please express your opinion also. You
come in the nick of time."
Then the crow laughed and said: "Gentlemen,
this is foolish. When you have eminent swans, pea-
cocks, nightingales, partridges, sheldrakes, pigeons,
cranes, and others, why anoint this ugly-faced fellow
who is blind in the daytime? It seems wrong to me.
For
Big hooked nose, and eyes asquint,
Ugly face without a hint
Of tenderness or beauty in 't.
Good-natured, it is fierce to see;
If he were mad, what might it be?
And furthermore:
Ugly, cruel, full of spleen,
Every word he speaks is mean;
If you make the owl your king,
You will fail in everything.
Besides, when Garuda is your king, what is this fellow
good for? Suppose he has virtue, still a second king
is not a good idea when you already have one. For
the saying runs:
A single king of lordly sway
Is good; but more than one will slay,
Like plural suns on Judgment Day.
Why, the very name of your genuine king keeps
others from taking liberties. As the proverb puts it:
Mere mention of a lordly monarch's name
To mean men, straightway saves from loss and shame.
And there is a saying:
The feigning of a great commission
Immensely betters your condition:
Feigning a message from the moon,
The rabbits dwelt in comfort soon."
"How was that?" asked the birds. And the crow
told
HOW THE RABBIT FOOLED THE ELEPHANT
In a part of a forest lived an elephant-king named
Four-Tusk, who had a numerous retinue of elephants.
His time was spent in protecting the herd.
Now once there came a twelve-year drought, so
that tanks, ponds, swamps, and lakes went dry. Then
all the elephants said to the lord of the herd: "O
King, our little ones are so tortured by thirst that
some are like to die, and some are dead. Pray devise a
method of removing thirst." So he sent in eight direc-
tions elephants fleet as the wind to search for water.
Now those who went east found beside a path
near a hermitage a lake named Lake of the Moon. It
was beautiful with swans, herons, ospreys, ducks,
sheldrakes, cranes, and water-creatures. It was em-
bowered in flowering sprays of branches drooping
under the weight of various blossoms. Both banks
were embellished with trees. It had beaches made
lovely by sheets of foam born of the splashing of
transparent waves that danced in the breeze and
broke on the shore. Its water was perfumed by the
ichor-juice that oozed from elephant-temples washed
clean of bees; for these flew up when the lordly crea-
tures plunged. It was ever screened from the heat of
the sun by hundreds of parasols in the shape of the
countless leaves of trees on its banks. It gave forth
deep-toned music from uncounted waves that turned
aside on meeting the plump legs, hips, and bosoms of
mountain maidens diving. It was brimming with
crystal water, and beautified with thickets of water-
lilies in full bloom. Why describe it? It was a seg-
ment of paradise.
When they saw this, they hastened back to report
to the elephant-king.
So Four-Tusk, on hearing their report, traveled
with them by easy stages to the Lake of the Moon.
And finding a gentle slope all around the lake, the ele-
phants plunged in, thereby crushing the heads, necks,
fore-paws and hind-paws of thousands of rabbits who
long before had made their home on the banks. Now
after drinking and bathing, the elephant-king with his
followers departed to his own portion of the jungle.
Then the rabbits who were left alive held an
emergency convention. "What are we to do now?"
said they. "Those fellows---curse their tracks!
will come here every day. Let some plan be framed
at once to prevent their return."
Thereupon a rabbit named Victory, perceiving
their terror and their utter woe at the crushing of
sons, wives, and relatives, said compassionately:
"Have no fear. They shall not return. I promise it.
For my guardian angel has granted me this grace."
And hearing this, the rabbit-king, whose name was
Block-Snout, said to Victory: "Dear friend, this is
beyond peradventure. For
Good Victory knows every fact
The textbooks teach; knows how to act
In every place and time. Where he
Is sent, there comes prosperity.
And again:
Speak for pleasure, speak with measure,
Speak with grammar's richest treasure,
Not too much, and with reflection
Deeds will follow words' direction.
The elephants, sir, making acquaintance with your
ripe wisdom, will become aware of my majesty, wis-
dom, and energy, though I am not present. For the
proverb says:
I learn if foreign kings be fools or no
By their dispatches or their nuncio.
And there is a saying:
The envoy binds; he loosens what is bound;
Through him success in war, if found, is found.
And if you go, it is as if I went myself.  Because, if
you
Speak what lies in your commission,
Speak with careful composition,
Grammar and good ethics seeking,
'Tis as if myself were speaking.
And again:
This is, in brief, the envoy's care:
An argument to fit the facts
And sound results, so far as speech
May be translated into acts.
"Depart then, dear friend. And may the office of
envoy prove a second guardian angel to you."
So Victory departed and espied the elephant-king
in the act of returning to the lake. He was surrounded
by thousands of lordly elephants, whose ears, like
flowering branches, were swaying in a dignified dance.
His body was dappled with masses of pollen from his
couch made of twigs from the tips of branches of
flowering cassia trees; so that he seemed a laden cloud
with many clinging lightning-flashes. His trumpeting
was as deep toned and awe inspiring as the clash of
countless thunderbolts from which in the rainy season
piercing flashes gleam. He had the glossy beauty of
leaves in a bed of pure blue lotuses. His twisting
trunk had the charm of a perfect snake. His presence
was that of an elephant of heaven. His two tusks,
shapely, smooth, and full, had the color of honey.
Around his entire visage rose a charming hum from
swarms of bees drawn by the fragrant perfume of the
ichor-juice that issued from his temples.
And Victory reflected: "It is impossible for folk
like me to come too near.  Because, as the proverb
puts it:
An elephant will kill you if
He touch; a serpent if he sniff;
King's laughter has a deadly sting;
A rascal kills by honoring.
I must by all odds seek impregnable terrain beforv
introducing myself."
After these reflections, he climbed upon a tall and
jagged rock-pile before saying: "Is it well with you,
lord of the two-tusked breed?" And the elephant-
king, hearing this, peered narrowly about, and saicf
"Who are you, sir?" "I am an envoy," said the rab-
bit. "In whose service?" asked the elephant, and the
envoy answered: "In the service of the blessed
Moon." "State your business," said the elephant-
king, and the rabbit stated it thus.
"You are aware, sir, that no injury may be done
an envoy in the discharge of his function.  For all
kings, without exception, use envoys as their mouth-
pieces.  Indeed, there is a proverb:
Though swords be out and kinsmen fall in strife,
The king still spares the harsh-tongued envoy's life.
"Therefore by command of the Moon I say to
you: 'Why, O mortal, why have you used violence
upon others, with no true reckoning of your own
power or your foe's? For the Scripture says:
All those who madly march to deeds,
Not reckoning who are masters,
Themselves or powerful enemies,
Are asking for disasters.
"'Now you have sinfully violated the Lake of the
Moon, known afar by my sacred name. And there
you have slain rabbits who are under my special
protection, who are of the race of that rabbit-king
cherished in my bosom. This is iniquitous. Nay, one
would think you the only creature in the world who
does not know the rabbit in the moon. But what is
gained by much speaking? Desist from such actions,
or great disaster will befall you at my hands. But
if from this hour you desist, great distinction will be
yours; for your body will be nourished by my moon-
light, and with your companions you shall pursue
your happy, carefree fancies in this forest. In the
alternative case, my light shall be withheld, your
body will be scorched by summer heat, and you with
your companions will perish.'"
On hearing this, the elephant-king felt his heart
stagger, and after long reflection he said: "It is true,
sir. I have sinned against the blessed Moon. Who
am I that I should longer contend with him? Pray
point out to me, and quickly, the way that I must
travel to win the blessed Moon's forgiveness/*
The rabbit said: "Come, sir, alone. I will point
it out.*' So he went by night to the Lake of the Moon,
and showed him the moon reflected in the water.
There was the brilliant, quivering disk, of lustrous
loveliness, surrounded by planets, the Seven Sages,
and hosts of stars, all dancing in the reflection of
heaven's broad expanse. And its circle was complete,
with the full complement of digits.
Seeing this, the elephant said: "I purify myself
and worship the deity," and he dropped upon the
water a trunk that two men's arms might have en-
circled. Thereby he disturbed the water, the moon's
disk danced to and fro as if mounted on a whirling
wheel, and he saw a thousand moons.
Then Victory started back in great agitation, and
said to the elephant-king: "Woe, woe to you, O King!
You have doubly enraged the Moon." The elephant
said: "For what reason is the blessed Moon angry
with me?"
"Because," said Victory, "you have touched this
water." So the elephant-king, with drooping ears,
bowed his head to the very earth in deep obeisance,
in order to win forgiveness from the blessed Moon.
And he spoke again to Victory: "My worthy sir, in
all other manners, also, beseech for me the forgiveness
of the blessed Moon. I shall never return here."
And with these words he went to his own place.
"And that is why I say:
The feigning of a great commission,�...
and the rest of it.
"But worse remains behind. The owl is a seedy
rascal, with a wicked soul. He could never protect
subjects. Or rather, to say nothing of protection,
you may anticipate actual danger from him. You
know the stanza:
A seedy umpire is not very
Pleasing to either adversary:
Rabbit and partridge teach you that
They died, confiding in the cat."
"How was that? Tell us about it/' said the birds,
and the crow told the story of
THE CATS JUDGMENT
At one time I was myself living in a certain tree.
And beneath the same tree dwelt another bird, a
partridge. So by virtue of our near neighborhood
there sprang up between us a firm friendship. Every
day after taking our meals and airings we spent the
evening hours in a round of amusements, such as
repeating witty sayings, telling tales from the old
story-books, solving puzzles and conundrums, or ex-
changing presents.
One day the partridge went foraging with other
birds to a spot where the rice was ripe and abundant,
and he did not return at nightfall. Of course, I missed
him greatly and I thought: "Alas! Why does not my
friend the partridge come home tonight? I am much
afraid he is caught in some trap, or has even been
killed." And many days passed while I grieved in
this way.
Now one evening a rabbit named Speedy made
himself at home in the partridge's old nest in the hole.
Nor did I say him nay, for I despaired of seeing the
partridge again.
However, one fine day the partridge, who had
grown extremely plump from eating rice, remembered
his old home and returned. This, indeed, is not to
be wondered at.
No mortal has such joy, although
In heaven's fields he roam,
As in his city, in his land,
And in his humble home.
Now when he saw the rabbit in the hole, he said
reproachfully: "Come now, rabbit, you have done a
shabby thing in occupying my apartment. Please
begone, and lose no time about it."
"You fool!" said the rabbit, "don't you know that
a dwelling is yours only while you occupy it?" "Very
well, then," said the partridge, "suppose we ask the
neighbors. For, to give you a legal quotation,

For ownership of cisterns, tanks,
Wells, groves, and houses, too,

The neighbors' testimony goes
Such is the legal view.

And again:

When house or field or well or grove

Or land is in dispute,
A neighbor's testimony is

Decisive of the suit."

Then the rabbit said: "You fool! Are you ignor-
ant of the consecrated tradition which says:

Suppose beside your neighbor you

For ten long years abide,
What weight have learned arguments?

Eyewitnesses decide.

Fool! Fool! Did you never hear the dictum of the
sage Narada?

The title to possession is

A ten years' habitation
With men. But with the birds and beasts

Mere present occupation.

"Hence, even supposing this apartment to be yours,
still it was unoccupied when I moved in, and now it is

mine."

"Well, well!" replied the partridge, "if you appeal
to consecrated tradition, come with me, and we will
consult the specialists. It shall be yours or mine
according to their decision." "Very well," said the
other, and together they started off to have their suit
decided. I, too, was at their heels, out of curiosity.
"I will just see what comes of all this," I said to my-
self.
Now they had not traveled far when the rabbit
asked the partridge: "My good fellow, who is to
pass judgment on our disagreement?" And the par-
tridge answered: "On a sand-bank by the sacred
Ganges---where there is sweet music from the dancing
waves that intercross and break when the water is
swept by nimble breezes---there dwells a tomcat
whose name is Curd-Ear. He abides unshaken in his
vow of penance and self-denial, and character has
begotten compassion."
But when the rabbit spied the cat, his soul stag-
gered with terror, and he said: "No, no! He is a
seedy rascal. You must have heard the proverb:
Oh, never trust a rogue for all
His pharisaic puzzling:
At holy shrines some saints are found
Quite capable of guzzling."
Upon hearing this, Curd-Ear, whose manner of
life had been assumed for the purpose of making an
easy livelihood, desired to win their confidence. He
therefore gazed straight at the sun, stood on his hind-
legs, lifted his fore-paws, blinked his eyes, and in
order to deceive them by pious sentiments, delivered
the following moral discourse. "Alas! Alas! All is
vanity. This fragile life passes in a moment. Union
with the beloved is an empty dream. Family endear-
ments are a conjurer's trick.  But for the moral law,
there would be no escape. Oh, listen to Scripture!
Each transitory day, O man,
To moral living give;
Else, like the blacksmith's bellows, you
Suck air, but do not live.
And furthermore:
Non-moral learning is a curse,
A dog's tail, nothing less,
That does not save from flies and fleas,
Nor cover nakedness.
And yet again:
A rotten ear among the wheat,
Among the birds a bat,
Is he who spurns the moral law;
The merest living gnat.
The flowers and fruit are better than the tree;
Better than curds is butter said to be;
Better than oil-cake, oil that trickles free;
Better than mortal man, morality.
The praise of constant steadfastness
Some wise professors sing;
But moral earnestness is swift,
Though many fetters cling.
Forget your prosings manifold;
The moral law is briefly told:
To help your neighbor---this is good;
To injure him is devilhood."
Having listened to this moral discourse, the rabbit
said: "Friend partridge, here on the river-bank is
the saint who expounds the moral law. Let us ask
him."
But the partridge said: "After all, he is our nat-
ural enemy. Let us ask him from a distance." So
together they began to question him: "O holy moral-
ist, a dispute has arisen between us. Pray give judg-
ment in accordance with the moral law. And which-
ever of us is found to speak falsely, him you may eat."
"Dear friends," said the cat, "I implore you not to
speak thus. My soul abhors every act of cruelty,
that street-sign pointing to hell. Surely, you know
the Scripture:
The holy first commandment runs
Not harsh, but kindly be
And therefore lavish mercy on
Mosquito, louse, and flea.
Why speak of hurting innocence?
For he, with purpose fell
Who injures even noxious beasts,
Is plunged in ghastly hell.
"Nay, even those who slay living creatures in the
act of sacrifice are befuddled, and their hermeneutic
theology is at fault. And if you object to me the
passage, 'One should sacrifice with goats/ in that
passage the word 'goats' signifies grain that has aged
seven years. 'Go, oats'---such is the true exegesis.
And then, consider the passage:
If he who cuts down trees or cattle,
Or makes a bloody slime in battle,
Should thereby win to heaven---well,
Who (let me ask you) goes to hell ?
"No, no. I shall eat nobody. However, I am
somewhat old and do not readily distinguish your
voices from a distance. So how am I to determine
winner and loser? In view of this, pray draw near
and make me acquainted with the case. Then I can
pronounce a judgment that discriminates the essence
of the matter, and thus causes no impediment in my
march to the other world. You know the stanza:
If any man, from pride or greed,
Timidity or wrath,
Judge falsely, he has set his foot
On hell's down-sloping path.
And again:
Who wrongs a sheep, slays kinsmen five;
Who wrongs a cow, slays ten;
A hundred die for maidens wronged;
A thousand die for men.
"Therefore confide in me and speak clearly at the
edge of my ear."
Why spin it out? That seedy rogue won their
trust so fully that both drew near him. Then, of
course, he seized them simultaneously, one with his
paw, the other with the saw of his teeth. And when
they were dead, he ate them both.
"And that is why I say:
A seedy umpire is not very. . . .
and the rest of it.
"Just so, you, too, being blind at night, if you take
as overlord this seedy fellow who is blind in the day-
time, will go the way of the rabbit and the partridge.
Reflect on this, then do what seems proper."
And all the birds, after listening to the crow's re-
marks, said: "He speaks well," and they flew to
their homes, planning to reassemble for consultation
on the question of a king. Only the owl remained
with his consort, for he was blind in the daytime.
There he sat in his chair of state, awaiting the anoint-
ing. And he called out: "Ho, there! Who takes my
orders? Why is the ceremony delayed?"
Thereupon his consort said: "My dear sir, the
crow has found means to hold up the ceremony. And
the birds have gone flying away. Only that crow,
for some reason or other, remains here all alone. Rise
at once, and I will conduct you home."
Then the owl was deeply disappointed, and he
said: "You monster! Why have you wronged me by
preventing the regal anointing? From this day there
is enmity between us. For the proverb says:
When arrows pierce or axes wound
A tree, it grows together sound;
From cruel, ugly speech you feel
A wound that time will never heal."
Thereupon he went home with his consort, while
the crow reflected: "Dear me! I have burdened my-
self with a needless enmity by speaking so. I should
have remembered:
All spoken words, if harsh and heedless
And inappropriate and needless,
Are self-condemnatory slips
That turn to poison on the lips.
And again:
However wise and strong you be,
Beware the needless enemy:
You would not swallow poison down
Because a doctor lives in town.
No man of sense vituperates
Another, while the public waits;
For even truth should be concealed,
If causing sorrow when revealed.
And finally:
Reflect with many a chosen friend;
Reflect alone, and to the end;
Then act. You are intelligent,
And fame's and wealth's recipient."
After these reflections, the crow also left the spot.
"For this cause, my son, we have an inherited
feud with the crows."
"Father," said Cloudy, "what should we do under
the circumstances?" And Live-Strong answered:
"Even in these circumstances there is an effective
procedure other than the six expedients. This I will
adopt, and will myself lead the way to conquer the
enemy. I will deceive them and put them in a fatal
situation. For the saying goes:
The strong, deft, clever rascals note,
Who robbed the Brahman of his goat."
"How was that?" asked Cloudy. And Live-Strong
told the story of
THE BRAHMAN'S GOAT
In a certain town lived a Brahman named Friend-
ly who had undertaken the labor of maintaining the
sacred fire. One day in the month of February, when
a gentle breeze was blowing, when the sky was veiled
in clouds and a drizzling rain was falling, he went to
another village to beg a victim for the sacrifice, and
said to a certain man: "O sacrificer, I wish to make
an offering on the approaching day of the new
moon. Pray give me a victim." And the man gave
him a plump goat, as prescribed in Scripture. This
he put through its paces, found it sound, placed it
on his shoulder, and started in haste for his own
city.
Now on the road he was met by three rogues whose
throats were pinched with hunger. These, spying the
plump creature on his shoulder, whispered together:
"Come now! If we could eat that creature, we should
have the laugh on this sleety weather. Let us fool
him, get the goat, and ward off the cold."
So the first of them changed his dress, issued from
a by-path to meet the Brahman, and thus addressed
that man of pious life: "O pious Brahman, why are
you doing a thing so unconventional and so ridicu-
lous? You are carrying an unclean animal, a dog, on
your shoulder. Are you ignorant of the verse:
The dog and the rooster,
The hangman, the ass,
The camel, defile you:
Don't touch them, but pass."
At that the Brahman was mastered by anger, and
he said: "Are you blind, man, that you impute dog-
hood to a goat?" "O Brahman," said the rogue, "do
not be angry. Go whither you will."
But when he had traveled a little farther, the
second rogue met him and said: "Alas, holy sir, alas!
Even if this dead calf was a pet, still you should not
put it on your shoulder. For the proverb says:
Touch not unwisely man or beast
That lifeless lie;
Else, gifts of milk and lunar fast
Must purify."
Then the Brahman spoke in anger: "Are you
blind, man? You call a goat a calf." And the rogue
said: "Holy sir, do not be angry. I spoke in ignorance.
Do as you will."
But when he had walked only a little farther
through the forest, the third rogue, changing his
dress, met him and said: "Sir, this is most improper.
You are carrying a donkey on your shoulder. Yet the
proverb tells you:
If you should touch an ass---be it
In ignorance or not
You needs must wash your clothes and bathe,
To cleanse the sinful spot.
Pray drop this thing, before another sees you."
So the Brahman concluded that it was a goblin in
quadruped form, threw it on the ground, and made
for home, terrified. Meanwhile, the three rogues
met, caught the goat, and carried out their plan.
"And that is why I say:
The strong, deft, clever rascals note, ....
and the rest of it.
"Moreover, there is sound sense in this:
Is any man uncheated by
New servants' diligence,
The praise of guests, the maiden's tears,
And roguish eloquence?
Furthermore, one should avoid a quarrel with a
crowd, though the individuals be weak. As the verse
puts it:
Beware the populace enraged;
A crowd's a fearsome thing:
The ants devoured the giant snake
For all his quivering."
"How was that?" asked Cloudy. And Live-
Strong told the story of
THE SNAKE AND THE ANTS
In a certain ant-hill lived a prodigious black
snake, and his name was Haughty. One day, instead
of following the beaten path out of his hole, he tried
.to crawl through a narrower crevice. In doing so, he
suffered a wound, because his body was huge, and the
opening was small, and fate willed it so.
Then the ants gathered about him, drawn by the
odor of blood from the wound, and drove him frantic.
How many did he kill? Or how many crush? Yet
their uncounted phalanx stung him in every member,
and enlarged the numerous wounds. And Haughty
perished.
"And that is why I say:
Beware the populace enraged, ....
and the rest of it.
"Furthermore, O King, I have something to tell
you, which you must consider, and ponder, and
do."
"Father," said Cloudy, "tell me what you have in
mind." And Live-Strong said: "Listen, my son. I
have discovered a fifth device, different from the
well-known four---conciliation, intrigue, bribery, and
fighting. And it is this. You must turn against me,
revile me with the hardest-hearted words you can
find, smear me with blood (which you will provide)
in order to deceive the enemy's spies, throw me out
at the foot of this banyan tree, and depart yourself
to Antelope Mountain. And there you must stay
with your retinue until by clever planning I win the
trust of all the enemy, discover the heart of their
fortress, and kill them---for they are blind in the day-
time. This plan I devised on the assumption that
their fortress is of simple construction, without egress
at the rear. For the saying goes:
A fort must have for egress, say
The specialists, a gap;
If this be lacking, it is not
A fortress, but a trap.
Nor should you feel any pity for me. For the proverb
says:
Pet and pamper servants well;
Love them as you love your life:
Yet consider them as dry
Tinder in the hour of strife.
Nor must you balk me in my design. For once more:
Cherish servants like yourself;
Guard them as you guard your life
Every day for one sole day,
When you meet your foe in strife."
With these words he started a sham fight with the
king. And Cloudy's retinue, seeing Live-Strong
jabber with unbridled license at the king, started up
to kill him. But Cloudy said: "Out of my path, you.
I take upon myself the chastisement of this traitorous
scoundrel." With this he pounced upon him, pecked
at him gently, smeared him with blood (which he had
provided), and departed with his retinue for Antelope
Mountain, as Live-Strong had recommended.
At this juncture the owl's consort, acting as spy
for the enemy, went and reported in detail to the
owl-king the disgrace of Cloudy's prime minister.
And the owl-king, informed of the occurrence, started
with his retainers at sundown on a crow-hunt. And
he said: "Hasten, friends, hasten! The enemy is
panic-stricken, is in full flight, and can be readily
caught. For the proverb says:
In flight, a fort becomes a trap
Where all defense is lacking;
'Tis easy then to beat a king
Whose men are busy packing."
With this battle-cry they flew to attack the ban-
yan tree. And failing to find a single crow, King Foe-
Crusher gleefully perched on a branch, and while the
court poets chanted flatteries, he gave orders: "Ho
there! Discover their line of retreat. Before they
establish themselves in a fort, I will be at their heels
and will kill them."
At this point Live-Strong reflected: "If the enemy
simply go home after learning what we have done, I
shall have accomplished nothing. For the proverb
says:
The first or second evidence
Of genuine intelligence
Is---leave a business unbegun,
Or, if begun, then see it done.
It would have been better not to undertake this than
to see the undertaking fail. I will reveal myself by
letting them hear me caw."
So he cawed with a feeble squeak. And the owls,
hearing this, started up to kill him. But Live-Strong
said: "Gentlemen, I am Cloudy's minister, Live-
Strong, reduced to this state by Cloudy himself.
Pray inform your own king. I have much to discuss
with him."
So the owl-king, informed by his followers, came,
beheld with astonishment the scars of many wounds,
and said: "Well, sir! How did you fall into this con-
dition? Tell me."
And Live-Strong said: "O King, listen. Yester-
day that rascal Cloudy, seeing how many crows you
had killed, was distracted by wrath and grief, and
started for your fortress. Whereupon I said: 'You
should not march against him. For they are strong,
and we are weak. Now the proverb advises those who
wish to thrive:
Do not, even in thought, offend
Stronger foes who will not bend;
They will feel no loss or shame;
You will die, a moth in flame.
You should seek peace by paying him tribute/ When
he heard this, he was made furious by rascally ad-
visers, suspected me of being a partisan of yours, and
reduced me to this state. Therefore your royal feet
are now my sole refuge. In a word, so long as I can
stir, I will conduct you to his abode, and cause the
total destruction of the crows."
On hearing this, Foe-Crusher took counsel with
the counselors who had served his father and his
grandfather. They were five in number, and their
names were Red-Eye, Fierce-Eye, Flame-Eye, Hook-
Nose, and Wall-Ear.
So first he questioned Red-Eye: "My worthy sir,
what is to be done under the circumstances?" And
Red-Eye said: "O King, what is there to consider
here? Kill him without hesitation. For the proverb
says:
Kill a weakling, lest he "grow
Hard to smite;
Later, with augmented power
He will fight.
Besides, you know how common people say: 'A lost
chance brings a curse/ And again:
He who will not when he may,
When he will, he shall have nay.
And this too:
The lighted funeral pile you may
Break up and fling apart;
But love, when torn and patched again,
Lives in an aching heart."
"How was that?" asked Foe-Crusher. And Red-
Eye told the story of
THE SNAKE WHO PAID CASH
There was once a Brahman in a certain place. His
time was wholly spent in unproductive farming.
Now one day, toward the end of summer, the heat
was too much for him, and he dozed in the shade of a
tree in the middle of his field. Not far away he saw,
peering over an ant-hill, a terrifying snake that thrust
forward a great, swelling hood. And he reflected:
"Surely, this is the guardian deity of the field, and I
never paid him honor. That is why my farm-work is
unproductive. I will pay him honor."
Thereupon he begged milk from somebody, put
it in a saucer, went to the ant-hill, and said: "O
guardian of the field! All this long time I did not
know that you were living here. Therefore I paid
you no honor. From now on, please be gracious to
me." With this he presented the milk and went home.
Now when he came back in the morning and
looked about, he found a gold dinar in the saucer. So
he went there every day alone, and offered milk, re-
ceiving a dinar each time. One day, however, the
Brahman went to town, instructing his son to carry
milk to the ant-hill. And the boy took the milk there,
set it down, and went home again.
The next day he went there, found a single dinar,
and thought: "Surely, this ant-hill is full of dinars. I
will kill that fellow and get them all." With this pur-
pose, while offering milk the next day, the Brahman's
boy struck the snake on the head with a cudgel. Yet
somehow---for fate willed it so---the snake did not die.
Instead, he furiously struck the boy with his sharp
fangs to such effect that the boy died at once. And
the relatives cremated him on a woodpile near the
field.
On the second day the father returned. And
learning from his relatives the cause of his son's
death, he found the facts as stated. And he said:
Be generous to all that lives;
Receive the needy guest:
If not, your own life fades away
Like swans from lotus nest.
"How was that?" asked the men. And the Brah-
man told the story of
THE UNSOCIAL SWANS
There was once a king named Gay-Chariot in a
certain place. He owned a lake named Lotus Lake,
which his soldiers guarded carefully. For many
golden swans lived there, and they gave one tail-
feather apiece every six months.
Now to that lake came a great bird, all of gold.
And they told him: "You cannot live among us.
For we have rented this lake at the rate of a tail-
feather for six months." And so, to cut a long story
short, a dispute arose.
Then the great bird sought the king's protection,
saying: "O King, those birds ask: 'What will our
king do? We give lodging to nobody/ And I said:
'You are not very polite. I will go and tell the king/
This is the situation. The king must decide."
Then the king said to his men: "Go, you. Kill
all the birds and bring them here at once." And they
started immediately, obeying the king's command.
Now one old bird saw the king's men with clubs in
their hands, and he said: "Well, kinsmen, this is
rather unpleasant. We must all hang together. Let
us fly up and away." And they did so.
"And that is why I say:
Be generous to all that lives, . . .
and the rest of it."
So in the morning the Brahman took milk again,
went to the spot, and called out, in an effort to win
the snake's confidence: "My son met the death that
suited his intelligence." Then the snake said:
The lighted funeral pile you may
Break up and fling apart;
But love, when torn and patched again,
Lives in an aching heart.
"Thus, when he is dead, you will without effort
enjoy a thornless kingdom."
Having listened to this proposal, the king asked
Fierce-Eye: "My worthy sir, what is your opinion?"
And Fierce-Eye said: "O King, his advice is heart-
less. For one does not kill a suppliant. No doubt you
have heard the old story:
The dove (there mentioned) entertained
His suppliant foeman slaughter-stained;
Paid honor due, his guest to greet;
And sacrificed himself for meat."
"How was that?" asked Foe-Crusher. And Fierce-
Eye told the story of
THE SELF-SACRIFICING DOVE
A ghastly fowler plied his trade
Of horror in a forest; made
All living creatures hold their breath:
He seemed to them the god of death.
He had no comrade on the earth,
No friend, no relative by birth.
They all renounced him; he had made
Them do so by his horrid trade.
For you know
The dreadful wretches bringing death
On those who love their living breath,
With natural repulsion (like
Fierce serpents) fill before they strike.
To snare, to imprison, and to drub
He took a net, a cage, a club,
And wandering daily in the wood,
He brought all creatures harm, not good.
While he was in the wood one day,
The sky grew black with clouds straightway;
So wild the wind, so fierce the rain,
It seemed the world dissolved in pain.
Then, as the heart within him quivered,
And every limb grew numb and shivered,
He sought where might a refuge be,
And chanced to come upon a tree.
Now as he rested, near and far
In sudden-clearing skies, each star
Shone bright; and he had wit to pray:
"O Lord, be kind to me today."
There was a dove upon the tree
Whose nest was in a cavity;
And since his wife was absent long,
He grieved for her in mournful song:
"The wind and rain were very great,
And my beloved wife is late
In coming home. When she is not
At home, home is an empty spot.
"The house is not the home; but where
The wife is found, the home is there.
The home without the wife is less
To me than some wild wilderness.
"Some wives their life's devotion give,
And in and for the husband live;
Whatever man has such a wife
Is heaped with blessings all his life."
From fowling-cage the female dove
Had caught the speech of grief and love;
And she was deeply gratified,
And to her husband thus replied:
"No woman earns the name of bride
Whose husband is not satisfied.
If he is happy, she may know
The gods she venerates are so.
"That woman should be burned entire
(Like vines that fade in forest-fire
While blossoms drop from clustered side)
Whose husband is not satisfied."
And she continued:
"Oh, harken heedfully, my dear;
My words are good for you to hear;
Though it should cost your life, defend
The guest who seeks in you a friend.
"Here lies a fowler; as a guest
He asks for comfort at your nest.
Since cold and hunger press him sore,
Begrudge him not from honor's store.
And the Scripture says:
"Whoever does not give his best
To cheer the late-arriving guest
Will see his merit borne away,
And for the other's sins will pay.
"Oh, let no hate against him rise
Who caged the wife you idolize;
It is my sins of former lives
That, fateful, hold me in the gyves.
For well you know:
"Disease, and poverty, and pain,
With woe that prison brings amain,
Are all the fruit of one sole tree,
Our own, our past iniquity.
"Abandon, therefore, thoughts of hate
Deriving from my captive state;
On virtue set your heart; and pay
This man such honor as you may."
On listening to his darling, who
Seemed virtue-woven through and through,
An unknown courage fired the dove;
He gave the fowler words of love.
"A hearty welcome, sir, to you;
What for your service may I do?
No more let anxious fancies roam,
For here with me you are at home."
In answer to his kindly words
Replied the murderer of birds:
"Well, dove, the cold is in me still;
Give me a remedy for chill."
The dove then brought a bonfire's sole
Surviving ember---one live coal,
And where a pile of dry leaves lay,
He kindled it to fire straightway.
"Now, sir, take heart; forgetting fear,
Resuscitate your members here;
Alas! I cannot put to flight
The cravings of your appetite.
"One patron feeds a thousand men;
One feeds a hundred; one feeds ten.
But I, whose virtue does not thrive,
Scarce keep my puny self alive.
"Ah, if you have not in your nest
Provision for a single guest,
Why occupy today, tomorrow
A nest that harbors naught but sorrow?
"I shall destroy my body, fain
To end its living with its pain,
That nevermore I stand confessed
Powerless to aid a needy guest."
And thus he blamed himself, you see;
The greedy fowler went scot-free:
Then---"I may yet your craving sate,
If one mere moment you will wait."
Whereat that creature free from sin,
Joy-quivering his soul within,
Walked round the fire, as it had been
His cherished home, and entered in.
When this the greedy fowler saw,
Compassion filled his soul, and awe.
He, while the dove was cooking, spoke
What from his heart a passage broke:
"None loves his soul, 'tis very plain,
Who smears it with a sinful stain.
The soul commits the sin; and late
Or soon, the soul must expiate.
"My thoughts are evil; my desire
Is ever set on what is dire:
It needs but little wit to tell
I steer my course for ghastly hell.
"A moral lesson let me draw
From what my savage spirit saw.
The high-souled dove, that I may eat,
Has sacrificed himself for meat.
"Henceforth let all enjoyment be
An unfamiliar thing to me;
I'll share the shallow water's fate
In August; will evaporate.
"Cold, wind, and heat I will embrace,
Grow thin and dirty, form and face,
Will fast by every method known,
Seek virtue, perfect and alone."
The fowler then apieces tore
Club, peg, net, cage---and what is more,
Set free the wretched female dove
Who sorrowed for her perished love.
But she, released from clutches dire,
Beheld her husband in the fire;
Whereat she gave expression so
To thoughts of horror and of woe:
"My lord! My love! What shall I do
With life that drags, apart from you?
What profit has a wretched wife,
Without a husband, of her life?
"For self-esteem, respect, and pride,
The family honor paid a bride,
Authority with all the brood
Of servants, die with widowhood."
Now after this lamenting sore,
This sorrow bitter evermore,
She went where lay her heart's desire,
Walked straight into the blazing fire.
And lo! She sees her husband shine
Oh, wonder!---in a car divine;
Her body wears a heavenly gown;
And heavenly gems hang pendent down.
While he, become a god, addressed
True consolation to her breast:
"The deed that you have done, is meet
In following your husband, sweet.
"There grow upon a man alive
Some thirty million hairs and five;
So many years in heaven spend
Wives following husbands to the end."
So he joyfully took her into the chariot, embraced
her, and lived happily. But the fowler sank into the
deepest despondency, and plunged into a great forest,
meditating death.
And there he saw a forest-fire
And entered it; for all desire
Was dead. His sins were burned away;
He went to heaven, there to stay.
"And that is why I say:
The dove (there mentioned) entertained, ....
and the rest of it."
Having listened to this, Foe-Crusher asked Flame-
Eye: "What is your opinion, sir, things standing as
they do?" And Flame-Eye said:
"She who always shrank from me
Hugs me to her breast.
Thank you, benefactor! Take
What you like the best."
And the thief replied:
"Nothing here that I should like;
Should I want a thing,
I'll return if she does not
Passionately cling."
"But," asked Foe-Crusher, "who is she that does
not cling? And who is the thief? I should like to hear
this one in detail." And Flame-Eye told the story of
THE OLD MAN WITH THE YOUNG WIFE
There was once an aged merchant in a certain
town, and his name was Lovelorn. To such an extent
had love clouded his reason that, when his wife died,
he gave much money in order to marry the daughter
of a penniless shopkeeper. But the girl was heart-
broken and could not bear to look at the old mer-
chant. This, indeed, might have been anticipated.
The silvered head will sue in vain,
A maiden's love beseeching;
The maid, despising it, is fain
To flee afar with screeching;
Like Hangman's Well it causes pain,
Where dead men's bones are bleaching.
And furthermore:
Slow, tottering steps the strength exhaust;
The eye unsteady blinks;
From driveling mouth the teeth are lost;
The handsome figure shrinks;
The limbs are wrinkled; relatives
And wife contemptuous pass;
The son no further honor gives
To doddering age. Alas!
Now one night, while she was turning her back to
him in bed, a thief entered the house. And she was
terrified at seeing a thief, and embraced her husband,
old as he was. He, for his part, felt every limb thrill
with astonishment and love, and he thought: "Gra-
cious me! Why does she hug me tonight?" Then,
peering narrowly about, he discovered the thief in a
corner, and reflected: "No doubt she embraces me
from fear of him." So he said to the thief:
"She who always shrank from me,
Hugs me to her breast;
Thank you, benefactor! Take
What you like the best."
And the thief made reply:
"Nothing here that I should like;
Should I want a thing,
I'll return if she does not
Passionately cling."
"Thus advantage may be anticipated from a
benefactor, thief though he be. How much more
from a suppliant guest? Besides, having been mal-
treated by them, he will labor for our success, or for
the revelation of their vulnerable point. In view of
this, he should not be killed."
Having listened to this view, Foe-Crusher ques-
tioned another counselor, namely, Hook-Nose. "My
worthy sir, what should be done under the present
circumstances?" And Hook-Nose answered: "O
King, he should not be killed. For
From enemies expect relief,
If discord pierce their host;
Thus, life was given by the thief
And cattle by the ghost."
"How was that?" asked Foe-Crusher. And Hook-
Nose told the story of
THE BRAHMAN, THE THIEF,
AND THE GHOST
There was once a poor Brahman in a certain place.
He lived on presents, and always did without such
luxuries as fine clothes and ointments and perfumes
and garlands and gems and betel-gum. His beard
and his nails were long, and so was the hair that
covered his head and his body. Heat, cold, rain, and
the like had dried him up.
Then someone pitied him and gave him two calves.
And the Brahman began when they were little and
fed them on butter and oil and fodder and other
things that he begged. So he made them very plump.
Then a thief saw them and the idea came to him
at once: "I will steal these two cows from this Brah-
man." So he took a rope and set out at night. But
on the way he met a fellow with a row of sharp teeth
set far apart, with a high-bridged nose and uneven
eyes, with limbs covered with knotty muscles, with
hollow cheeks, with beard and body as yellow as a
fire with much butter in it.
And when the thief saw him, he started with acute
fear and said: "Who are you, sir?"
The other said: "I am a ghost named Truthful.
It is now your turn to explain yourself."
The thief said: "I am a thief, and my acts are
cruel. I am on my way to steal two cows from a poor
Brahman."
Then the ghost felt relieved and said: "My dear
sir, I take one meal every three days. So I will just
eat this Brahman today. It is delightful that you and
I are on the same errand."
So together they went there and hid, waiting
for the proper moment. And when 'the Brahman
went to sleep, the ghost started forward to eat him.
But the thief saw him and said: "My dear sir, this is
not right. You are not to eat the Brahman until I
have stolen his two cows."
The ghost said: "The racket would most likely
wake the Brahman. In that case all my trouble would
be vain."
"But, on the other hand/' said the thief, "if any
hindrance arises when you start to eat him, then I
cannot steal the two cows either. First I will steal
the two cows, then you may eat the Brahman."
So they disputed, each crying "Me first! Me
first!" And when they became heated, the hubbub
waked the Brahman. Then the thief said: "Brah-
man, this is a ghost who wishes to eat you." And the
ghost said: "Brahman, this is a thief who wishes to
steal your two cows."
When the Brahman heard this, he stood up and
took a good look. And by remembering a prayer to
his favorite god, he saved his life from the ghost,
then lifted a club and saved his two cows from the
thief.
"And that is why I say:
From enemies expect relief,
and the rest of it. Besides:
The Scriptures tell a holy tale
Of sacrificial love,
How Shibi gave the hawk his flesh
As ransom for the dove
showing that it is contrary to religion to slay a sup-
pliant."
Having listened to this opinion, the king asked
Wall-Ear: "What is your view, sir? Tell me." And
Wall-Ear said: "O King, he certainly should not be
killed. For if you spare his life, you two may well
grow fond of each other, and spend the time pleasant-
ly. There is a saying:
Be quick with mutual defense
In honest give-and-take;
Or perish, like the ant-hill beast
And like the belly-snake."
"How was that?" asked Foe-Crusher. And Wall-
Ear told the story of
THE SNAKE IN THE PRINCE'S BELLY
In a certain city dwelt a king whose name was
Godlike. He had a son who wasted daily in every
limb because of a snake that used his belly as a home
instead of an ant-hill. So the prince became dejected
and went to another country. In a city of that coun-
try he begged alms, spending his time in a great
temple.
Now in that city was a king named Gift, who had
two daughters in early womanhood. One of these
bowed daily at her father's feet with the greeting:
"Victory, O King," while the other said: "Your
deserts, O King."
At this the king grew angry, and said: "See,
counselors. This young lady speaks malevolently.
Give her to some foreigner. Let her have her own
deserts/' To this the counselors agreed, and gave
the princess, with very few maid-servants, to the
prince who made his home in the temple.
And she was delighted, accepted her husband like
a god, and went with him to a far country. There
by the edge of a tank in a distant city she left the
prince to look after the house while she went with
her maids to buy butter, oil, salt, rice, and other
supplies. When her shopping was done, she returned
and found the prince with his head resting on an ant-
hill. And from his mouth issued the head of a hooded
snake, taking the air. Likewise another snake crawled
from the ant-hill, also to take the air.
When these two saw each other, their eyes grew
red with anger, and the ant-hill snake said: "You
villain! How can you torment in this way a prince
who is so perfectly handsome?" And the snake in the
prince's mouth said: "Villain yourself! How can
you bemire those two pots full of gold?" In this
fashion each laid bare the other's weakness.
Then the ant-hill snake continued: "You villain!
Doesn't anybody know the simple remedy of drinking
black mustard and so destroying you?" And the
belly-snake retorted: "And doesn't anybody know
the simple way to destroy you, by pouring in hot
water?"
Now the princess, hiding behind a branch, over-
heard their conversation, and did just as they sug-
gested. So she made her husband sound and well, and
acquired vast wealth. When she returned to her own
country, she was highly honored by father, mother,
and relatives, and lived happily. For she had her
deserts.
"And that is why I say:
Be quick with mutual defense, ....
and the rest of it."
Now Foe-Crusher, having heard their advice,
agreed. But Red-Eye, perceiving that the matter
was decided, continued his remarks with a quiet
sneer: "Alas! Alas! Our lord the king has been wick-
edly done to death by you gentlemen. For the prov-
erb says:
Where honor is withheld or paid
Mistakenly, 'tis clear
Three things have unrestricted course:
Famine, and death, and fear.
And again:
It argues utter want of sense
To pardon obvious offense:
The carpenter upon his head
Took wife and him who fouled his bed."
"How was that?" asked the counselors, and Red-
Eye told the story of
THE GULLIBLE CARPENTER
There was once a carpenter in a certain village.
His wife was a whore, and reputed to be such. So he,
desiring to test her, thought: "How can I put her
to the test? For the proverb says:
Fire chills, rogues bless, and moonlight burns
Before a wife to virtue turns.
"Now I know from popular gossip that she is un-
faithful. For the saying goes:
All things that are not seen or heard
In science or the Sacred Word,
All things in interstellar space
Are known among the populace."
After these reflections, he said to his wife: "To-
morrow morning, my dear, I am going to another
village, where I shall be detained several days. Please
put me up a nice lunch." And her heart quivered
when she heard this; she eagerly dropped everything
to make delicious dishes, almost pure butter and
sugar. In fact, the old saw was justified:
When lowering clouds
Shut in the day,
When streets are mired
With sticky clay,
When husband lingers
Far away,
The flirt becomes
Supremely gay.
Now at dawn the carpenter rose and left his
house. When she had made sure that he was gone,
with laughing countenance she spent the dragging
day in trying on all her best things. Then she called
on an old lover and said: "My husband has gone to
another village---the rascal! Please come to our house
when the people are asleep." And he did so.
Now the carpenter spent the day in the forest,
stole into his own house at twilight by a side entrance,
and hid under the bed. At this juncture the other
fellow arrived and got into bed. And when the car-
penter saw him, his heart was stabbed by wrath,
and he thought: "Shall I rise and smite him? Or
shall I wait until they are asleep and kill them both
without effort? Or again, shall I wait to see how she
behaves, listen to what she says to him?" At this
moment she softly locked the door and went to bed.
But as she did so, she stubbed her toe on the
carpenter's body. And she thought: "It must be
that carpenter---the rascal!---who is testing me. Well,
I will give him a taste of woman's tricks."
While she was thinking, the fellow became insis-
tent. But she clasped her hands and said: "Dear
and honored sir, you must not touch me." And he
said: "Well, well! For what purpose did you invite
me?"
"Listen," said she. "I went this morning to
Gauri's shrine to see the goddess. There all at once
I heard a voice in the sky, saying: 'What am I to do,
my daughter? You are devoted to me, yet in six
months' time, by the decree of fate, you will be a
widow.' Then I said: 'O blessed goddess, since you
are aware of the calamity, you also know the remedy.
Is there any means of making my husband live a
hundred years?' And the goddess replied: 'Indeed
there is---a remedy depending on you alone/ Of
course I said: 'If it cost my life, pray tell me, and I
will do it/ Then the goddess said: 'If you go to bed
with another man, and embrace him, then the un-
timely death that threatens your husband will pass
to him. And your husband will live another hundred
years/ For this purpose I invited you. Now do what
you had in mind. The words of a goddess must not
be falsified---so much is certain/' Then his face
blossomed with noiseless laughter, and he did as she
said.
Now the carpenter, fool that he was, felt his body
thrill with joy on hearing her words, and he issued
from under the bed, saying: "Bravo, faithful wife!
Bravo, delight of the family! Because my heart was
troubled by the gossip of evil creatures, I pretended
a trip to another village in order to test you, and
lay hidden under the bed. Come now, embrace
me!"
With these words he embraced her and lifted her
to his shoulder, then said to the fellow: "My dear
and honored sir, you have come here because my
good deeds earned this happiness. Through your
favor I have won a full hundred years of life. You,
too, must mount my shoulder/'
So he forced the fellow, much against his will, to
mount his shoulder, and then went dancing about
to the doors of the houses of all his relatives.
"And that is why I say:
It argues utter want of sense
To pardon obvious offense, . . . .
and the rest of it.
"We are certainly uprooted and undone. For the
proverb is right in saying:
Shrewd men unmask a foe
Who seems a friend,
Whose speech is kind, whose acts
To hatred tend.
And again:
Before fools' counsel flees
Prosperity, though won;
Its place and time are lost,
Like dark before the sun."
But they all disregarded his advice, picked Live-
Strong up, and started to carry him to their fortress.
And on the journey Live-Strong said: "O King, I
have done nothing yet, and I am in a sad state. Why
are you so kind to me? Nay, I desire to enter the
blazing fire. Pray put me under obligations by pro-
viding fire."
Now Red-Eye pierced his purpose and said:
"Why do you wish to enter fire?" And Live-Strong
replied: "For your sake I have been plunged into
this calamity by Cloudy. Therefore I wish to be re-
born as an owl in order to requite their enmity." Now
Red-Eye, being a master of diplomacy, rejoined:
"My dear sir, you are wily and plausible. Even if
reborn as an owl, you would highly esteem your
corvine provenience. There is a story that illustrates
the point:
Though mountain, sun, and cloud, and wind
Were suitors at her feet,
The mouse-maid turned a mouse again
Nature is hard to beat."
"How was that?" asked Live-Strong. And Red-
Eye told the story of
MOUSE-MAID MADE MOUSE
The billows of the Ganges were dotted with pearly
foam born of the leaping of fishes frightened at hear-
ing the roar of the waters that broke on the rugged,
rocky shore. On the bank was a hermitage crowded
with holy men devoting their time to the performance
of sacred rites---chanting, self-denial, self-torture,
study, fasting, and sacrifice. They would take puri-
fied water only, and that in measured sips. Their
bodies wasted under a diet of bulbs, roots, fruits, and
moss. A loin-cloth made of bark formed their scanty
raiment.
The father of the hermitage was named Yajna-
valkya. After he had bathed in the sacred stream and
had begun to rinse his mouth, a little female mouse
dropped from a hawk's beak and fell into his hand.
When he saw what she was, he laid her on a banyan
leaf, repeated his bath and mouth-rinsing, and per-
formed a ceremony of purification. Then through the
magic power of his holiness, he changed her into a girl,
and took her with him to his hermitage.
As his wife was childless, he said to her: "Take
her, my dear wife. She has come into life as your
daughter, and you must rear her carefully." So the
wife reared her and spoiled her with petting. As soon
as the girl reached the age of twelve, the mother saw
that she was ready for marriage, and said to her
husband: "My dear husband, how can you fail to
see that the time is passing when your daughter
should marry ?"
And he replied: "You are quite right, my dear.
The saying goes:
Before a man is gratified,
These gods must treat her as a bride
The fire, the moon, the choir of heaven;
In this way, no offense is given.
Holiness is the gift of fire;
A sweet voice, of the heavenly choir;
The moon gives purity within:
So is a woman free from sin.
Before nubility, 'tis said
That she is white; but after, red;
Before her womanhood is plain,
She is, though naked, free from stain.
The moon, in mystic fashion, weds
A maiden when her beauty spreads;
The heavenly choir, when bosoms grow;
The fire, upon the monthly flow.
To wed a maid is therefore good
Before developed womanhood;
Nor need the loving parents wait
Beyond the early age of eight.
The early signs one kinsman slay;
The bosom takes the next away;
Friends die for passion gratified;
The father, if she ne'er be bride.
For if she bides a maiden still,
She gives herself to whom she will;
Then marry her in tender age:
So warns the heaven-begotten sage.
If she, unwed, unpurified,
Too long within the home abide,
She may no longer married be:
A miserable spinster, she.
A father then, avoiding sin,
Weds her, the appointed time within
(Where'er a husband may be had)
To good, indifferent, or bad.
Now I will try to give her to one of her own station.
You know the saying:
Where wealth is very much the same,
And similar the family fame,
Marriage (or friendship) is secure;
But not between the rich and poor.
And finally:
Aim at seven things in marriage;
All the rest you may disparage:
"But
Get money, good looks,
And knowledge of books,
Good family, youth,
Position, and truth.
"So, if she is willing, I will summon the blessed
sun, and give her to him." "I see no harm in that,"
said his wife. "Let it be done."
The holy man therefore summoned the sun, who
appeared without delay, and said: "Holy sir, why
am I summoned?" The father said: "Here is a
daughter of mine. Be kind enough to marry her."
Then, turning to his daughter, he said: "Little girl,
how do you like him, this blessed lamp of the three
worlds?" "No, father," said the girl. "He is too
burning hot. I could not like him. Please summon
another one, more excellent than he is."
Upon hearing this, the holy man said to the sun:
"Blessed one, is there any superior to you?" And
the sun replied: "Yes, the cloud is superior even to
me. When he covers me, I disappear."
So the holy man summoned the cloud next, and
said to the maiden: "Little girl, I will give you to
him." "No," said she. "This one is black and frigid.
Give me to someone finer than he."
Then the holy man asked: "O cloud, is there any-
one superior to you ?" And the cloud replied: "The
wind is superior even to me."
So he summoned the wind, and said: "Little girl,
I give you to him." "Father/* said she, "this one is
too fidgety. Please invite somebody superior even to
him." So the holy man said: "O wind, is there any-
one superior even to you ?" "Yes," said the wind.
"The mountain is superior to me."
So he summoned the mountain and said to the
maiden: "Little girl, I give you to him." "Oh,
father," said she. "He is rough all over, and stiff.
Please give me to somebody else."
So the holy man asked: "O kingly mountain, is
there anyone superior even to you ?" "Yes," said the
mountain. "Mice are superior to me."
Then the holy man summoned a mouse, and pre-
sented him to the girl, saying: "Little girl, do you
like this mouse?"
The moment she saw him, she felt: "My own
kind, my own kind," and her body thrilled and quiv-
ered, and she said: "Father dear, turn me into a
mouse, and give me to him. Then I can keep house
as my kind of people ought to do."
And her father, through the magic power of his
holiness, turned her into a mouse, and gave her to
him.
"And that is why I say:
Though mountain, sun, and cloud, and wind, . � � .
and the rest of it."
But they paid no heed to Red-Eye's reasoning,
and took the crow to their fortress, to the destruction
of their race.   And on  the journey Live-Strong
laughed in his heart and said:
The secrets of diplomacy
To him alone were plain
Who, instant in his master's cause,
Advised that I be slain.
"Now if they were to take his advice, not even
the slightest misfortune would befall them."
When they came to the fortress gate, Foe-Crusher
said: "Come, my friends! Give this Live-Strong
whatever chamber he prefers---for he wishes us well."
And Live-Strong, hearing this, reflected: "I must
now devise a plan for their destruction. This is not
possible if I live in their midst. For they would ob-
serve motions betraying my purpose, and would
keep their eyes open. Only by remaining near the
gate can I accomplish my desire."
He therefore said to the owl-king: "O King, what
the king has said, is eminently right. Yet I, too, am a
student of diplomacy and a well-wisher. I know that
even one who is loyal and pure in purpose should not
dwell in the heart of a fortress. I will therefore take
my place here at the fortress gate and pay daily
homage, my body sanctified by the dust from your
lotus feet."
To this the owl-king agreed, and his efficient
caterers daily gave Live-Strong, by special command
of the king, the pick of the viands. So that in a very
few days he grew strong as a peacock.
But Red-Eye, seeing how Live-Strong was being
pampered, was amazed, and he said to the counselors
and to the king himself: "Dear me! These counselors
are a pack of fools, and you, too, sir. I cannot think
otherwise. Then there is the saying:
I played the fool at first; then he
Who had me on his tether;
And then the king and counselor
We all were fools together."
"How was that?" they asked. And Red-Eye told
the story of
THE BIRD WITH GOLDEN DUNG
There was once a great tree on a mountain side.
On it lived a bird in whose dung gold appeared.
One day a hunter came to the spot, and directly
in front of him the bird dropped its dung, which at
the moment of falling turned to gold. At this the
hunter was amazed.
"Well, well!" said he. "For eighty years, man and
boy, I have had bird-trapping on the brain, and I
never once saw gold in a bird's dung." So he set a
snare in the tree. And the bird, fool that he was,
forgot the danger, and perched on the customary
spot. Of course, he was caught immediately.
Then the hunter freed him from the snare, put
him in a cage and took him home. But he reflected:
"What am I to do with this bird of ill omen? If any-
body should ever discover his peculiarity, it would be
reported to the king. In that case my very life would
be in genuine danger. I will take the bird and report
to the king myself." And he did so.
Now when the king saw the bird, his lotus eyes
blossomed and he felt supremely gratified. "Come
now, guardsmen," said he. "Look after this bird
with anxious care. Give him everything he wants to
eat and drink."
Then a counselor said: "He was hatched from
an egg. Why keep him ? You have no evidence save
the mere incredible assurance of a hunter. Is gold
ever present in bird-dung? Take this bird from the
cage and set him free."
So the king, taking the counselor's advice, freed
the bird, who perched on the lofty arch of the door-
way long enough to drop dung which was of gold.
Then he recited the stanza:
I played the fool at first; then he
Who had me on his tether;
And then the king and counselor
We all were fools together.
After which he took his carefree flight through the
atmosphere.
"And that is why I say:
I played the fool at first, ....
and the rest of it."
But once more---for fate was hostile---they neg-
lected Red-Eye's counsel, sound as it was, and pam-
pered Live-Strong further with varied viands, in-
cluding plenty of meat.
Then Red-Eye called together his personal ad-
herents, and said to them privately: "The end is at
hand. The welfare of our king, and his fortress, are
things of the past. I have given him such counsel as
an ancestral counselor should give. Let us now, for
our part, seek another fortress in the mountains. For
the saying goes:
Joy comes from knowing what to dread,
And sorrow smites the dunderhead:
A long life through, the woods I've walked,
But never heard a cave that talked."
"How was that?" they asked. And Red-Eye told
the story of
THE CAVE THAT TALKED
There was once a lion in a part of a forest, and
his name was Rough-Claw. One day he found nothing
whatever to eat in his wanderings, and his throat was
pinched by hunger. At sunset he came to a great
mountain cave and went in, for he thought: "Surely,
some animal will come into this cave during the night.
I will hide and wait."
Presently the owner of the cave, a jackal named
Curd-Face, came to the door and began to sing:
"Cave ahoy! Cave aho-o-oy!" Then after a mo-
ment's silence, he continued in the same tone:
"Hello! Don't you remember how you and I made
an agreement that I was to speak to you when I came
back from the world outside, and that you were to
sing out to me? But you won't speak to me today.
So I am going off to that other cave, which will return
my greeting/'
Now when he heard this, the lion thought: "I
see. This cave always calls out a greeting when the
fellow returns. But today, from fear of me, it doesn't
say a word. This is natural enough. For
The feet and hands refuse to act
When peril terrifies;
A trembling seizes every limb;
And speech unuttered dies.
"I will myself call out a greeting, which he will
follow to its source, so providing me with a dinner."
The lion thereupon called out a greeting. But the
cave so magnified the roar that its echo filled the
circuit of the horizon, thus terrifying other forest
creatures as well, even those far distant. Meanwhile
the jackal made off, repeating the stanza:
Joy comes from knowing what to dread,
And sorrow smites the dunderhead:
A long life through, the woods I've walked^
But never heard a cave that talked.
"Take this to heart and come with me." And
Red-Eye, having made his decision, departed for
another fortress, accompanied by a retinue of fol-
lowers.
At Red-Eye's departure, Live-Strong was over-
joyed. And he reflected: "Very good, indeed. Red-
Eye's flight is a blessing to us. For he was farsighted,
while the rest are numskulls. I can easily destroy
them now. For the proverb says:
If no farsighted counselors,
Long-tried, secure,
Aid him, the downfall of a king
Is swift and sure.
And there is sound reasoning in this:
The shrewd discover enemies
Disguised as friends
In senseless counselors whose speech
To evil tends."
After these reflections, he dropped each day one
fagot from the forest into his own nest, with the ul-
timate purpose of setting the cave afire. Nor did the
owls, poor fools, perceive that he was building up his
nest in order to burn them alive. Well, there is sense
in the saying:
Cause your friends no bitter woes;
Do not fraternize with foes:
Friends, when lost, are friends no more;
Enemies were lost before.
Thus, pretending to build a nest, Live-Strong con-
structed a woodpile at the fortress gate. Then at
sunrise, when the owls became blind, he hastened
away and reported to Cloudy: "My lord and king,
I have prepared the enemy's cave for burning. Come
with your retainers, each bringing a lighted fagot
from the forest, to throw on my nest at the gate of the
cave. Thus all your foes will die in torments like those
in Pot-baking Hell."
At this Cloudy was delighted and said: "Father,
tell me your adventures. It is long since we met."
"No, my son," said Live-Strong. "This is no time
for talk. Some enemy spy might possibly report my
journey hither. And our blind enemy, thus informed,
might make his escape. Make haste, make haste.
For the proverb says:
When speed is needful, ne'er permit
Delay, but do it pat;
Else, wrathful gods are sure to strike
The undertaking flat.
And again:
Whatever deed you have in mind
(Especially when fate is kind),
Do quickly. If you wait a bit,
Then time will suck the juice of it.
"Later, when your enemies are slain, and you
have returned to your home, I will tell the whole
story in carefree humor."
So Cloudy and his followers, taking Live-Strong's
advice, seized one lighted fagot apiece in their bills,
flew to the gate of the cave, and threw their fagots
upon Live-Strong's nest. Then all the owls (being
blind in the daytime) remembered Red-Eye's counsels
as they suffered the torments of Pot-baking Hell. In
this fashion Cloudy exterminated his foes and re-
turned to his old fortress in the banyan tree.
There he mounted the lion-throne and, his heart
overflowing with joy, he questioned Live-Strong in
full session of his court: "Father, how did you pass
the time in the midst of the enemy? For the proverb
Better a plunge in blazing fire
(The righteous know)
Than momentary contact with
A wicked foe."
And Live-Strong said: "My lord and king!
Whatever path provides escape
When danger's face is seen,
With clear decision follow, if
It noble seem, or mean:
Two arms like trunks of elephants,
Fight-calloused, skilled to wield
The bow of heaven, Arjun felt
To woman's bracelets yield.
The wise and strong, awaiting days
More prosperous, must grant
Obedience to wicked lords
Whose speech is adamant:
Gigantic Bhima, smoke-begrimed,
Puffing at labor, and
A ladle flourished in his fist,
Was cook in Matsya land.
The prudent, hopeful man should act
As suits an evil case,
Should steel his heart to carry through
A holy deed, or base:
Great Arjun with a calloused arm
From twanging bow divine
Effeminately danced, and saw
His tinkling girdle shine.
The wise, alert, ambitious man,
If he expect success,
Must wait on fortune, watch his step,
And curb his stateliness:
Yudhishthir King, with pilgrim's staff,
Long drew his painful breath,
Though worshiped by his brothers, great
As War, and Wealth, and Death.
So Kunti's handsome, powerful twins,
High birth writ on their brows,
Were menials at Virata's court,
And lived by counting cows.
So queenly Draupadi, with youth's
And matchless beauty's seal,
In charm most like a goddess, fell
By turn of fortune's wheel;
And haughty maidens called her slave
And sneered at her for sport,
What time she powdered sandalwood
In Matsya's royal court."
"Father," said Cloudy, "this dwelling with an
enemy seems to me like the sword-blade ordeal."
"So it is," said Live-Strong. "But I never saw such a
pack of fools anywhere. Not one was sensible except
Red-Eye. He, indeed, has great capacity, an intelli-
gence not blunted by his extensive scientific attain-
ments. He discovered my exact purpose. But as for
the other counselors, they were great fools, making
a living by a mere pretense of giving good counsel,
with no flair for verity. They were not even aware
of this:
'Tis ruinous to trust the scamps
Who come to you from hostile camps;
Such rivals you should chase away,
For constant trouble does not pay.
The foeman serving as a scout,
Who knows (by bobbing in and out)
Your favored chair, familiar bed,
And how you drink, and what you're fed,
Your travels to another town
Will strike his heedless foeman down.
The prudent therefore guards himself
The source of virtue, love, and pelf
With every effort, strain, and stress:
For death will follow heedlessness.
And there is plenty of sense in this:
Who, ill-advised, does not commit
Grave faults of savoirfaire?
What glutton has not much unrest
Within himself to bear?
Whom does not fortune render proud?
Whom does not death lay low?
To whom do not possessions bring
Abundant harm and woe?
The steady forfeit glory, while
The restless forfeit friends;
The bankrupt forfeits family,
The banker, better ends;
The man of passion forfeits books,
The fawner, friendship's flower;
The king with careless counselors
Must forfeit kingly power.
"Yes, O King, I have experienced in person what
you were kind enough to put into words: that associ-
ation with the enemy is equal to the sword-blade
ordeal. As the old verse puts it:
Bear even foes upon your back;
When fortune clogs
Your path, endure. The great black snake
Slew many frogs."
"How was that?" asked Cloudy. And Live-
Strong told the story of
THE FROGS THAT RODE SNAKEBACK
There was once an elderly black snake in a certain
spot, and his name was Slow-Poison. He considered
the situation from this point of view: "How in the
world can I get along without overtaxing my ener-
gies?" Then he went to a pond containing many
frogs, and behaved as if very dejected.
As he waited thus, a frog came to the edge of the
water and asked: "Uncle, why don't you bustle about
today for food as usual ?"
"My dear friend," said Slow-Poison, "I am afflict-
ed. Why should I wish for food? For this evening,
as I was bustling about for food, I saw a frog and
made ready to catch him. But he saw me and, fearing
death, he escaped among some Brahmans intent upon
holy recitation, nor did I perceive which way he went.
But in the water at the edge of the pond was the
great toe of a Brahman boy, and stupidly deceived
by its resemblance to a frog, I bit it, and the boy died
immediately. Then the sorrowing father cursed me
in these terms: 'Monster! Since you bit my harmless
son, you shall for this sin become a vehicle for frogs,
and shall subsist on whatever they choose to allow
you/ Consequently, I have come here to serve as
your vehicle."
Now the frog reported this to all the others. And
every last one of them, in extreme delight, went and
reported to the frog-king, whose name was Water-
Foot. He in turn, accompanied by his counselors,
rose hurriedly from the pond---for he thought it an
extraordinary occurrence---and climbed upon Slow-
Poison's hood. The others also, in order of age,
climbed on his back. Yet others, finding no vacant
spot, hopped along behind the snake. Now Slow-
Poison, with an eye to making his living, showed
them fancy turns in great variety. And Water-Foot,
enjoying contact with his body, said to him:
I'd rather ride Slow-Poison than
The finest horse I've seen,
Or elephant, or chariot,
Or man-borne palanquin.
The next day, Slow-Poison was wily enough to
move very slowly. So Water-Foot said: "My dear
Slow-Poison, why don't you carry us nicely, as you
did before?"
And Slow-Poison said: "O King, I have no carry-
ing power today because of lack of food." "My dear
fellow," said the king, "eat the plebeian frogs."
When Slow-Poison heard this, he quivered with
joy in every member and made haste to say: "Why,
that is a part of the curse laid on me by the Brahman.
For that reason I am greatly pleased at your com-
mand." So he ate frogs uninterruptedly, and in a
very few days he grew strong. And with delight and
inner laughter he said:
The trick was good. All sorts of frogs
Within my power have passed.
The only question that remains,
Is: How long will they last?
Water-Foot, for his part, was befooled by Slow-
Poison's plausibilities, and did not notice a thing.
At this moment another black snake, a tremen-
dous fellow, arrived on the scene. And being amazed
at the sight of Slow-Poison used as a vehicle by frogs,
he said: "Partner, they are our natural food, yet
they use you as a vehicle. This is repellent." And
Slow-Poison said:
I know I should not cany frogs;
I have it well in mind;
But I am marking time, as did
The Brahman butter-blind.
"How was that?" asked the snake. And Slow-
Poison told the story of
THE BUTTER-BLINDED BRAHMAN
There was once a Brahman named Theodore in a
certain town. His wife, being unchaste and a pur-
suer of other men, was forever making cakes with
sugar and butter for a lover, and so cheating her
husband.
Now one day her husband saw her and said: "My
dear wife, what are you cooking? And where are you
forever carrying cakes? Tell the truth."
But her impudence was equal to the occasion, and
she lied to her husband: "There is a shrine of the
blessed goddess not far from here. There I have
undertaken a fasting ceremony, and I take an offer-
ing, including the most delicious dishes." Then she
took the cakes before his very eyes and started for
the shrine of the goddess, imagining that after her
statement, her husband would believe it was for the
goddess that his wife was daily providing delicious
dishes. Having reached the shrine, she went down
to the river to perform the ceremonial bath.
Meanwhile her husband arrived by another road
and hid behind the statue of the goddess. And his
wife entered the shrine after her bath, performed the
various rites---laving, anointing, giving incense,
making an offering, and so on---bowed before the god-
dess, and prayed: "O blessed one, how may my
husband be made blind?"
Then the Brahman behind the goddess* back
spoke, disguising his natural tone: "If you never stop
giving him such food as butter and butter-cakes, then
he will presently go blind."
Now that loose female, deceived by the plausible
revelation, gave the Brahman just that kind of food
every day. One day the Brahman said: "My dear,
I don't see very well." And she thought: "Thank
the goddess."
Then the favored lover thought: "The Brahman
has gone blind. What can he do to me ?" Whereupon
he came daily to the house without hesitation.
But at last the Brahman caught him as he entered,
seized him by the hair, and clubbed and kicked him
to such effect that he died. He also cut off his wicked
wife's nose, and dismissed her.
"And that is why I say:
I know I should not carry frogs ....
and the rest of it."
Then Slow-Poison, with noiseless laughter,
hummed over the verse:
The trick was good. All sorts of frogs ....
and the rest of it. And Water-Foot, hearing this, was
conscience stricken, and wondering what he meant,
inquired: "My dear sir, what do you mean by re-
citing that repulsive verse?" "Nothing at all," said
Slow-Poison, desiring to mask his purpose. And
Water-Foot, befooled by his plausible manner, failed
to perceive his treachery.
Why spin it out? He ate them all so completely
that not even frog-seed was left.
"And that is why I say:
Bear even foes upon your back, ....
and the rest of it. Thus, O King, just as Slow-Poison
destroyed the frogs through the power of intelligence,
so did I destroy all the enemy. There is much wisdom
in this:
The forest-fire leaves roots entire,
Though trunks remain a shell;
The flooding pool of water cool
Uproots the roots as well."
"Very true," said Cloudy. "And besides:
This is the greatness of the great
Whom gems of wisdom decorate;
Despite what hurts and hinders, too,
They see an undertaking through."
"Very true," said Live-Strong. "And once again:
The final penny of a debt,
The final foeman dire,
The final twinges of disease,
The final spark of fire
Finality on these imposed
Leaves nothing to desire.
"O King, you are truly fortunate. For your under-
taking has had final success. Indeed, valor is not
sufficient to end a matter. Victory is wisdom's busi-
ness. As the proverb says:
Tis not the sword destroys a foe,
'Tis wit that utterly lays low:
Swords kill the body; wit destroys
Fame, family, and regal joys.
"Thus, success comes with minimum effort to a
man of wisdom and manliness. For
Wisdom broods o'er the inception;
Memory does not fail;
Means appear to predilection;
Counsels wise prevail;
Sparkles fruitful meditation;
Mind attains its height;
Joy achieves its consummation
In a worthy fight.
"Thus kingship belongs to the man possessing
prudence, capacity for self-sacrifice, and courage. As
the verse puts it:
Associate in full delight
With someone who is wise,
Self-sacrificing, brave; thereby
Win virtue as a prize;
On virtue follows money; and
On money follows fame;
Then, personal authority;
And then, the kingly name."
And Cloudy replied: "It is wonderful how im-
mediate is the reward of knowing social ethics. By
virtue of which you penetrated and exterminated
Foe-Crusher with his retinue." Whereupon Live-
Strong said:
"Where at last you need sharp measures,
First try gentle measures there:
Thus the lofty, lordly tree-trunk
Is not felled without a prayer.
"And yet, O my king, why say of a future matter
either that it involves no efFort or that it is not readily
attainable? There is wisdom in the saying:
Since words with actions fail to suit,
The timidly irresolute
Who see a thousand checks and blocks
Turn into public laughingstocks.
Nor are thoughtful men heedless even in minor
matters. For
The negligent who say:
'Some day, some other day
The thing is petty, small;
Demands no thought at all/
Are, heedless, headed straight
For that repentant state
That ever comes too late.
"But as for my master, who has overcome his
foes, he may sleep tonight as soundly as ever he did.
You know the saying:
In houses where no snakes are found,
One sleeps; or where the snakes are bound:
But perfect rest is hard to win
With serpents bobbing out and in.
"And again:
A noble purpose to attain
Desiderates extended pain,
Asks man's full greatness, pluck, and care,
And loved ones aiding with a prayer.
Yet if it climb to heart's desire,
What man of pride and fighting fire,
Of passion, and of self-esteem
Can bear the unaccomplished dream?
His heart indignantly is bent
(Through its achievement) on content.
"Therefore my heart is at peace. For I saw the
undertaking through. Therefore may you now long
enjoy this kingdom without a thorn---intent on the
safeguarding of your people---your royal umbrella,
throne, and glory unshaken through the long succes-
sion of son, grandson, and beyond. Remember:
A king should bring his people ease,
But he should also aim to please;
His reign is else of little note,
A neck-teat on a female goat.
And once again:
Love of virtue, scorn of vice,
Wisdom---make a kingdom's price.
Then is Glory proud as slave,
Then her plumes and pennons brave
Near the white umbrella wave.
"Nor must you, in the thought, 'My kingdom is
won/ shatter your soul with the intoxication of
glory. And this because the power of kings is a thing
uncertain. Kingly glory is hard to climb as a bam-
boo-stem; hard to hold, being ready to tumble in a
moment, with whatever effort it be held upright;
even though conciliated, yet sure to slip away at last;
fidgety as the bandar-log; unequilibrated as water on
a lotus-leaf; mutable as the wind's path; untrust-
worthy as rogues' friendship; hard to tame as a ser-
pent; gleaming but a moment like a strip of evening
cloud; fragile by nature, like the bubbles on water;
ungrateful as the substance of man's body; lost in
the moment of attainment, like the treasure of a
dream. And furthermore:
Whenever kings anointed are,
Let wit spy trouble from afar;
Anointing-jars too often spill,
With holy water, pending ill.
"And no man in the wide world is beyond the
clutch of pending ill. As the poet sings:
Remember Rama, wandering far;
Remember Nala's sinking star;
With Bali's bonds, the Vrishnis' tomb,
And Lanka's monster-monarch's doom;
The Pandus' forest-borne disaster,
And knightly Arjun, dancing-master.
Time brings us woe in countless shapes.
What savior is there? Who escapes?
Ah, where is Dasharath, who rose to heaven
And dwelt its king beside?
Ah, where King Sagar, he to whom 'twas given
To bind the ocean's tide?
Where arm-born Prithu? Where is Manu gone,
Sun-child (yet suns still rise) ?
Imperious Time awakened them at dawn,
At evening closed their eyes.
And again:
Where is Mandhatar, conqueror supreme?
Where Satyavrat, the king?
God-ruling Nahush? Keshav, e'er the gleam
Of science following?
They and their lordly elephants, I ween,
Their cars, their heavenly throne,
By lofty Time conferred, in Time were seen,
And lost through Time alone.
And yet again:
The king, his counselors,
His maidens gay,
His golden groves, Fate stings.
They sink away.
"Thus, having won kingly glory, quivering like
the ear of a rogue elephant, take delight in her, but
trust in wisdom only/'
Here ends Book III, called "Crows and Owls,"
which treats of peace, war, and the other four expedi-
ents. The first verse runs:
Reconciled although he be,
Never trust an enemy.
For the cave of owls was burned,
When the crows with fire returned.


\end{document}

%%% Local Variables:
%%% mode: latex
%%% TeX-master: t
%%% End:
