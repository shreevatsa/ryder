\documentclass[draft]{book}
\usepackage{alltt}
\usepackage{verse}
\newcommand{\sk}[1]{\textit{#1}}
%\newcommand{\attrib}[1]{\rightline{#1}}
\newcommand{\attrib}[1]{%
\nopagebreak{\raggedleft\footnotesize #1\par}}
% {\catcode`\^^M=13%
%   \gdef\myobeylines{\catcode`\^^M=13 \def^^M{\par\noindent}}%
% }
%\newenvironment{pverse}{\begin{verse}\parskip=0pt plus 1.0 pt \myobeylines}{\end{verse}}
\newenvironment{pverse}[1][0]{\begin{verse}\indentpattern{#1}\begin{patverse*}}{\end{patverse*}\end{verse}}
%\newenvironment{pverse}{\begin{quote}\begin{alltt}\normalfont}{\end{alltt}\end{quote}}
%\newenvironment{pverse}{\begin{center}\begin{minipage}{0.75\textwidth}\begin{alltt}\normalfont}{\end{alltt}\end{minipage}\end{center}}
\setlength{\beforepoemtitleskip}{0.0pt}
\settowidth{\beforepoemtitleskip}{}
\setlength{\afterpoemtitleskip}{0.0pt}

\newcommand{\story}[1]{\centerline{#1}}

\hyphenation{Pancha-tantra}
\title{The Panchatantra}
\author{Arthur W. Ryder (Tr.)}
\date{}


\begin{document}
%\maketitle

This is the esteemed translation of the \sk{Panchatantra}, by Arthur
William Ryder [1877--1938], based on Dr. Johannes Hertel's text.  Originally
published 1925 by the University of Chicago.  As the copyright term in
India is 60 years past death of author, this work passed into the
public domain in India in 1998.  Copyright laws may be different in
your country.  In particular, it is still under copyright in the
United States, as of 2008.

It is also available under the Digital Library of India Project, at
dli.iiit.ac.in

\attrib{---R. Shreevatsa}

\newpage

\part{Introductions}

\chapter{Translator's Introduction}

%TODO: Figure out how to get these centred
\section{I}

%TODO: Figure out how to get these centred
\begin{pverse}
One Vishnusharman, shrewdly gleaning\\
  All worldly wisdom's inner meaning,\\
In these five books the charm compresses\\
  Of all such books the world possesses.\\
\end{pverse}
 \attrib{---\textsc{Introduction to the} \sk{Panchatantra}}

The \sk{Panchatantra} contains the most widely known stories in the
world. If it were further declared that the \sk{Panchatantra} is the
best collection of stories in the world, the assertion could hardly be
disproved, and would probably command the assent of those possessing
the knowledge for a judgment. Assuming varied forms in their native
India, then traveling in translations, and translations of
translations, through Persia, Arabia, Syria, and the civilized
countries of Europe, these stories have, for more than twenty
centuries, brought delight to hundreds of millions.

Since the stories gathered in the \sk{Panchatantra} are very ancient,
and since they can no longer be ascribed to their respective authors,
it is not possible to give an accurate report of their genesis, while
much in their subsequent history will always remain obscure.
Dr. Hertel, the learned and painstaking editor of the text used by the
present translator, believes that the original work was composed in
Kashmir, about 200 B.C. At this date, however, many of the individual
stories were already ancient. He then enumerates no less than
twenty-five recensions of the work in India.  The text here translated
is late, dating from the year 1199 A.D.

It is not here intended to summarize the history of these stories in
India, nor their travels through the Near East and through Europe. The
story is attractive---whose interest is not awakened by learning, for
example, that in this work he makes the acquaintance of one of La
Fontaine's important sources? Yet here, as elsewhere, the work of the
``scholars'' has been of somewhat doubtful value, diverting attention
from the primary to the secondary, from literature itself to facts,
more or less important, about literature. The present version has not
been made by a scholar, but by the opposite of a scholar, a lover of
good books, eager, so far as his powers permit, to extend an accurate
and joyful acquaintance with the world's masterpieces. He will
therefore not endeavor to tell the history of the \sk{Panchatantra},
but to tell what the \sk{Panchatantra} is.

\section{II}
\begin{pverse}[001]
Whoever learns the work by heart,\\
Or through the story-teller's art\\
  Becomes acquainted,\\
His life by sad defeat---although\\
The king of heaven be his foe---\\
  Is never tainted.
\end{pverse}
\attrib{---\textsc{Introduction to the} \sk{Panchatantra}}


The \sk{Panchatantra} is a \sk{niti-shastra}, or textbook of
\sk{niti}. The word \sk{niti} means roughly ``the wise conduct of
life.''  Western civilization must endure a certain shame in realizing
that no precise equivalent of the term is found in English, French,
Latin, or Greek.  Many words are therefore necessary to explain what
\sk{niti} is, though the idea, once grasped, is clear, important, and
satisfying.

First of all, niti presupposes that one has considered, and rejected,
the possibility of living as a saint. It can be practiced only by a
social being, and represents an admirable attempt to answer the
insistent question how to win the utmost possible joy from life in the
world of men.

The negative foundation is security. For example,
if one is a mouse, his dwelling must contain recesses
beyond the reach of a cat's paw. Pleasant stanzas
concerning the necessity of security are scattered
throughout the work. Thus:
\begin{pverse}
The poor are in peculiar need\\
Of being secret when they feed;\\
The lion killed the ram who could\\
Not check his appetite for food.
\end{pverse}
or again:
\begin{pverse}
In houses where no snakes are found,\\
One sleeps; or where the snakes are bound:\\
But perfect rest is hard to win\\
With serpents bobbing out and in.
\end{pverse}

The mere negative foundation of security requires
a considerable exercise of intelligence, since the world
swarms with rascals, and no sensible man can imagine
them capable of reformation.
\begin{pverse}
Caress a rascal as you will,\\
He was and is a rascal still:\\
All salve- and sweating-treatments fail\\
To take the kink from doggy's tail.
\end{pverse}
Yet roguery can be defeated; for by its nature it is
stupid.
\begin{pverse}
Since scamp and sneak and snake\\
So often undertake\\
A plan that does not thrive,\\
The world wags on, alive.
\end{pverse}
Having made provision for security, in the realization that
\begin{pverse}
A man to thrive\\
Must keep alive,
\end{pverse}
one faces the necessity of having money. The \sk{Panchatantra}, being
very wise, never falls into the vulgar error of supposing money to be
important. Money must be there, in reasonable amount, because it is
unimportant, and what wise man permits things unimportant to occupy
his mind? Time and again the \sk{Panchatantra} insists on the misery
of poverty, with greatest detail in the story of ``Gold's Gloom'' in the
second book, never perhaps with more point than in the stanza:
\begin{pverse}
A beggar to the graveyard hied\\
And there ``Friend corpse, arise,'' he cried;\\
``One moment lift my heavy weight\\
Of poverty; for I of late\\
Grow weary, and desire instead\\
Your comfort; you are good and dead.''\\
The corpse was silent. He was sure\\
'Twas better to be dead than poor.
\end{pverse}

Needless to say, worldly property need not be, indeed should not be,
too extensive, since it has no value in possession, but only in use:
\begin{pverse}
In case of horse or book or sword,\\
Of woman, man or lute or word,\\
The use or uselessness depends\\
On qualities the user lends.
\end{pverse}

Now for the positive content of \sk{niti}. Granted se- curity and
freedom from degrading worry, then joy results from three
occupations---from resolute, yet circumspect, use of the active
powers; from intercourse with like-minded friends; and above all, from
worthy exercise of the intelligence.

Necessary, to begin with, for the experience of true joy in the world
of men, is resolute action. The difficulties are not blinked:
\begin{pverse}
There is no toy\\
Called easy joy;\\
But man must strain\\
To body's pain.
\end{pverse}
Time and again this note is struck---the difficulty and
the inestimable reward of sturdy action. Perhaps the
most splendid expression of this essential part of \sk{niti}
is found in the third book, in the words which the
crow, Live-Strong, addresses to his king, Cloudy:
\begin{pverse}
A noble purpose to attain\\
Desiderates extended pain,\\
Asks man's full greatness, pluck, and care,\\
And loved ones aiding with a prayer.\\
Yet if it climb to heart's desire,\\
What man of pride and fighting fire,\\
Of passion and of self-esteem\\
Can bear the unaccomplished dream?\\
His heart indignantly is bent\\
(Through its achievement) on content.\\
\end{pverse}

Equal stress is laid upon the winning and holding of intelligent
friends. The very name of the second book is ``The Winning of Friends'';
the name of the first book is ``The Loss of Friends.'' Throughout the
whole work, we are never permitted to be long oblivious of the rarity,
the necessity, and the pricelessness of friendship with the
excellent. For, indeed,
\begin{pverse}
The days when meetings do not fail\\
With wise and good\\
Are lovely clearings on the trail\\
Through life's wild wood.
\end{pverse}
So speaks Slow, the turtle; and Swift, the crow, expresses it thus:
\begin{pverse}
They taste the best of bliss, are good,\\
And find life's truest ends,\\
Who, glad and gladdening, rejoice\\
In love, with loving friends.
\end{pverse}

Last of all, and in a sense including all else, is the use of the
intelligence. Without it, no human joy is possible, nothing beyond
animal happiness.

\begin{pverse}
For if there be no mind\\
Debating good and ill,\\
And if religion send\\
No challenge to the will,\\
If only greed be there\\
For some material feast,\\
How draw a line between\\
The man-beast and the beast?
\end{pverse}
One must have at disposal all valid results of scholarship, yet one must not be a scholar. For
\begin{pverse}
Scholarship is less than sense;\\
Therefore seek intelligence.
\end{pverse}
One must command a wealth of detailed fact, ever alert to the
deceptiveness of seeming fact, since often-times
\begin{pverse}
The firefly seems a fire, the sky looks flat;\\
Yet sky and fly are neither this nor that.
\end{pverse}
One must understand that there is no substitute for judgment, and no
end to the reward of discriminating judgment:
\begin{pverse}
To know oneself is hard, to know\\
Wise effort, effort vain;\\
But accurate self-critics are\\
Secure in times of strain.
\end{pverse}
One must be ever conscious of the past, yet only as it offers material
for wisdom, never as an object of brooding regret:
\begin{pverse}
For lost and dead and past\\
The wise have no laments:\\
Between the wise and fools\\
Is just this difference.
\end{pverse}

This is the lofty consolation offered by a woodpecker to a hen-sparrow
whose eggs have been crushed by an elephant with the spring fever. And
the whole matter finds its most admirable expression in the noble
words of Cheek, the jackal:
\begin{pverse}
What is learning whose attaining\\
Sees no passion wane, no reigning\\
Love and self-control?\\
Does not make the mind a menial,\\
Finds in virtue no congenial\\
Path and final goal?\\
Whose attaining is but straining\\
For a name, and never gaining\\
Fame or peace of soul?
\end{pverse}

This is \sk{niti}, the harmonious development of the powers of man, a
life in which security, prosperity, resolute action, friendship, and
good learning are so combined as to produce joy. It is a noble ideal,
shaming many tawdry ambitions, many vulgar catch-words of our day. And
this noble ideal is presented in an artistic form of perfect fitness,
in five books of wise and witty stories, in most of which the actors
are animals.

\section{III}
\begin{pverse}
Better with the learned dwell,\\
Even though it be in hell\\
Than with vulgar spirits roam;\\
Palaces that gods call home.
\end{pverse}
\attrib{---\sk{Panchatantra}, Book II}

The word \sk{Panchatantra} means the ``Five Books,'' the Pentateuch. Each of
the five books is independent, consisting of a framing story with
numerous inserted stories, told, as fit circumstances arise, by one or
another of the characters in the main narrative. Thus, the first book
relates the broken friendship of the lion Rusty and the bull Lively,
with some thirty inserted stories, told for the most part by the two
jackals, Victor and Cheek. The second book has as its framing story
the tale of the friendship of the crow, the mouse, the turtle, and the
deer, whose names are Swift, Gold, Slow, and Spot. The third book has
as framing story the war between crows and owls.

These three books are of considerable length and show great skill in
construction. A somewhat different impression is left by Books IV and
V. The framing story of Book IV, the tale of the monkey and the
crocodile, has less interest than the inserted stories, while Book V
can hardly be said to have a framing story, and it ends with a couple
of grotesque tales, somewhat different in character from the others.
These two shorter books, in spite of the charm of their contents, have
the appearance of being addenda, and in some of the older recensions
are reduced in bulk to the verge of extinction.

The device of the framing story is familiar in oriental works, the
instance best known to Europeans being that of the Arabian
Nights. Equally characteristic is the use of epigrammatic verses by
the actors in the various tales. These verses are for the most part
quoted from sacred writings or other sources of dignity and
authority. It is as if the animals in some English beast-fable were to
justify their actions by quotations from Shakespeare and the
Bible. These wise verses it is which make the real character of the
\sk{Panchatantra}. The stories, indeed, are charming when regarded as pure
narrative; but it is the beauty, wisdom, and wit of the verses which
lift the \sk{Panchatantra} far above the level of the best story-books.  It
hardly needs to be added that in the present version, verse is always
rendered by verse, prose by prose. The titles of the individual
stories, however, have been supplied by the translator, since the
original has none.

The large majority of the actors are animals, who have, of course, a
fairly constant character. Thus, the lion is strong but dull of wit,
the jackal crafty, the heron stupid, the cat a hypocrite. The animal
actors present, far more vividly and more urbanely than men could do,
the view of life here recommended---a view shrewd, undeceived, and
free of all sentimentality; a view that, piercing the humbug of every
false ideal, reveals with incomparable wit the sources of lasting joy.

\rightline{\textsc{Arthur W. Ryder}}
\noindent
\parbox{7in}{
Berkeley, California\\
July, 1925
}


\end{document}
