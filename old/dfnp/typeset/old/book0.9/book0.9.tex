\documentclass[draft]{book}
\usepackage{alltt}
\usepackage{verse}
\newcommand{\sk}[1]{\textit{#1}}
%\newcommand{\attrib}[1]{\rightline{#1}}
\newcommand{\attrib}[1]{%
\nopagebreak{\raggedleft\footnotesize #1\par}}
% {\catcode`\^^M=13%
%   \gdef\myobeylines{\catcode`\^^M=13 \def^^M{\par\noindent}}%
% }
%\newenvironment{pverse}{\begin{verse}\parskip=0pt plus 1.0 pt \myobeylines}{\end{verse}}
\newenvironment{pverse}[1][0]{\begin{verse}\indentpattern{#1}\begin{patverse*}}{\end{patverse*}\end{verse}}
%\newenvironment{pverse}{\begin{quote}\begin{alltt}\normalfont}{\end{alltt}\end{quote}}
%\newenvironment{pverse}{\begin{center}\begin{minipage}{0.75\textwidth}\begin{alltt}\normalfont}{\end{alltt}\end{minipage}\end{center}}
\setlength{\beforepoemtitleskip}{0.0pt}
\settowidth{\beforepoemtitleskip}{}
\setlength{\afterpoemtitleskip}{0.0pt}

\newcommand{\story}[1]{\centerline{#1}}

\hyphenation{Pancha-tantra}
\title{The Panchatantra}
\author{Arthur W. Ryder (Tr.)}
\date{}


\begin{document}

\chapter{Introduction}

\begin{pverse}
One Vishnusharman, shrewdly gleaning\\
All worldly wisdom's inner meaning,\\
In these five books the charm compresses\\
Of all such books the world possesses.
\end{pverse}
And this is how it happened.

In the southern country is a city called Maidens'
Delight. There lived a king named Immortal-Power.
He was familiar with all the works treating of the wise
conduct of life. His feet were made dazzling by the
tangle of rays of light from jewels in the diadems of
mighty kings who knelt before him. He had reached
the far shore of all the arts that embellish life. This
king had three sons. Their names were Rich-Power,
Fierce-Power, Endless-Power, and they were supreme
blockheads.

Now when the king perceived that they were
hostile to education, he summoned his counselors and
said: ``Gentlemen, it is known to you that these sons
of mine, being hostile to education, are lacking in discernment.
So when I behold them, my kingdom
brings me no happiness, though all external thorns are
drawn. For there is wisdom in the proverb:

\begin{pverse}
  Of sons unborn, or dead, or fools,\\
  Unborn or dead will do:\\
  They cause a little grief, no doubt;\\
  But fools, a long life through.
\end{pverse}

And again:

\begin{pverse}
  To what good purpose can a cow\\
  That brings no calf nor milk, be bent?\\
  Or why beget a son who proves\\
  A dunce and disobedient?
\end{pverse}

Some means must therefore be devised to awaken their intelligence.''

And they, one after another, replied: ``O King, first one learns
grammar, in twelve years. If this subject has somehow been mastered,
then one masters the books on religion and practical life. Then the
intelligence awakens.''

But one of their number, a counselor named Keen,
said: ``O King, the duration of life is limited, and the
verbal sciences require much time for mastery.
Therefore let some kind of epitome be devised to
wake their intelligence. There is a proverb that says:

\begin{pverse}
  Since verbal science has no final end,\\
  Since life is short, and obstacles impend,\\
  Let central facts be picked and firmly fixed,\\
  As swans extract the milk with water mixed.
\end{pverse}

``Now there is a Brahman here named Vishnusharman, with a reputation
for competence in numerous sciences. Intrust the princes to him. He
will certainly make them intelligent in a twinkling.''

When the king had listened to this, he summoned Vishnusharman and
said: ``Holy sir, as a favor to me you must make these princes
incomparable masters of the art of practical life. In return, I will
bestow upon you a hundred land-grants.''

And Vishnusharman made answer to the king: ``O King, listen. Here is
the plain truth. I am not the man to sell good learning for a hundred
land-grants.  But if I do not, in six months' time, make the boys
acquainted with the art of intelligent living, I will give up my own
name. Let us cut the matter short.  Listen to my lion-roar. My
boasting arises from no greed for cash. Besides, I have no use for
money; I am eighty years old, and all the objects of sensual desire
have lost their charm. But in order that your request may be granted,
I will show a sporting spirit in reference to artistic matters. Make a
note of the date. If I fail to render your sons, in six months' time,
incomparable masters of the art of intelligent living, then His
Majesty is at liberty to show me His Majestic bare bottom.''

When the king, surrounded by his counselors, had listened to the
Brahman's highly unconventional promise, he was penetrated with
wonder, intrusted the princes to him, and experienced supreme content.
Meanwhile, Vishnusharman took the boys, went home, and made them learn
by heart five books which he composed and called: (I) ``The Loss of
Friends,'' (II) ``The Winning of Friends,'' (III) ``Crows and Owls,'' (IV)
``Loss of Gains,'' (V) ``Ill-considered Action.''

These the princes learned, and in six months' time they answered the
prescription. Since that day this work on the art of intelligent
living, called \sk{Panchatantra}, or the ``Five Books,'' has traveled the
world, aiming at the awakening of intelligence in the young. To sum
the matter up:

\begin{pverse}[001]
Whoever learns the work by heart,\\
Or through the story-teller's art\\
     Becomes acquainted,\\
His life by sad defeat---although\\
The king of heaven be his foe---\\
     Is never tainted.
\end{pverse}


\end{document}

%%% Local Variables:
%%% mode: latex
%%% TeX-master: t
%%% End:
