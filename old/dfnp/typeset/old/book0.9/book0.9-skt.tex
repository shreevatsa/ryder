\documentclass{article}

\usepackage{fontspec}
\setmainfont{Devanagari MT}

\begin{document}


कथा-मुखम्

ऒम्̣ नमः श्री-शारदा-गणपति-गुरुभ्यः | महा-कविभ्यॊ नमः |

ब्रह्मा रुद्रः कुमारॊ हरि-वरुण-यमा वह्निर् इंद्रः कुबॆरश्
चंद्रादित्यौ सरस्वत्य्-उदधि-युग-नगा वायुर् उर्वी-भुजंगाः |
सिढा नद्यॊ श्विनौ श्रीर् दितिर् अदिति-सुता मातरश् चंडिकाद्या
वॆदास् तीर्थानि यक्षा गण-वसु-मुनयः पांतु नित्यम्̣ ग्रहाश् च ||

\begin{verse}
मनवॆ वाचस्पतयॆ शुक्राय पराशराय स-सुताय |
चाणक्याय च विदुषॆ नमॊ स्तु नय-शास्त्र-कर्त्ड़्भ्यः || ।१||
\end{verse}

\begin{verse}
सकलार्थ-शास्त्र-सारम्̣ जगति समालॊक्य विष्णुशर्मॆदम् |
तंत्रैः पञ्चभिर् ऎतच् चकार सुमनॊहरम्̣ शास्त्रम् || ।२||
\end{verse}

तद् यथानुश्रूयतॆ | अस्ति दक्षिणात्यॆ जनपदॆ महिलारॊप्यम्̣ नाम नगरम् | तत्र सकलार्थि-सार्थ-कल्प-द्रुमः प्रवर-ंड़्प-मुकुट-मणिम् अजरीचयचर्चितचरण-युगलः सकल-कल्प-पारंगतॊ मरशक्तिर् नाम राजा बभूव | तस्य त्रयः पुत्राः परम-दुर्मॆधसॊ वसुशक्तिर् उग्रशक्तिर् अनॆकशक्तिश् चॆति नामानॊ बभूवुः |

अथ राजा तान् शास्त्र-विमुखान् आलॊक्य सचिवान् आहूय प्रॊवाच-भॊः ज्ञातम् ऎतद् भवद्भिर् यन् ममैतॆ त्रयॊ पि पुत्राः शास्त्र-विमुखा विवॆक-हीनाश् च | तद् ऎतान् पश्यतॊ मॆ महद् अपि राज्यम्̣ न सौख्यम् आवहति | अथवा साध्व् इदम् उच्यतॆ-

अजात-म्ड़्त-मूर्खॆभ्यॊ म्ड़्ताजातौ सुतौ वरम् |
यतस् तौ स्वल्प-दुःखाय यावज्-जीवम्̣ जडॊ दहॆत् || ।३||

वरम्̣ गर्भ-स्रवॊ वरम् ड़्तुषु नैवाभिगमनम्
वरम्̣ जातः प्रॆतॊ वरम् अपि च कंयैव जनिता |
वरम्̣ बंध्या भार्या वरम् अपि च गर्भॆषु वसतिर्
न चाविदग्धान् रूप-द्रविण-गुण-युक्तॊ पि तनयः || ।४||

किम्̣ तया क्रियतॆ धॆंवा या न सूतॆ न दुग्धदा |
कॊ र्थः पुत्रॆण जातॆन यॊ न विद्वान् न भक्तिमान् || ।५||

तद् ऎतॆषाम्̣ यथा बुढि-प्रबॊधनम्̣ भवति तथा कॊ प्य् उपायॊ नुष्ठीयताम् | अत्र च मद्-दत्ताम्̣ व्ड़्त्तिम्̣ भुञ्जानानाम्̣ पंडितानाम्̣ पञ्चशती तिष्ठति | ततॊ यथा मम मनॊरथाः सिढिम्̣ यांति तथानुष्ठीयताम् इति |

तत्रैकः प्रॊवाच-दॆव द्वादशभिर् वर्षैर् व्याकरणम्̣ श्रूयतॆ | ततॊ धर्म-शास्त्राणि मंव्-आदीनि अर्थ-शास्त्राणि चाणक्यादीनि काम-शास्त्राणि वात्स्यायनादीनि | ऎवम्̣ च ततॊ दर्मार्थ-काम-शास्त्राणि ज्ञायंतॆ | ततः प्रतिबॊधनम्̣ भवति |

अथ तन्-मध्यतः सुमतिर् नाम सचिवः प्राह-अशाश्वतॊ यम्̣ जीवितव्य-विषयः | प्रभूत-काल-ज्ञॆयानि शब्द-शास्त्राणि | तत् संक्षॆप-मात्रम्̣ शास्त्रम्̣ किञ्चिद् ऎतॆषाम्̣ प्रबॊधनार्थम्̣ चिंत्यताम् इति | उक्तम्̣ च यतः-

अनंतपारम्̣ किल शब्द-शास्त्रम्
स्वल्पम्̣ तथायुर् बहवश् च विघ्नाः |
सारम्̣ ततॊ ग्राह्यम् अपास्य फल्गु
हम्̣सैर् यथा क्षीरम् इवांबुध्यात् || ।६||

तद् अत्रास्ति विष्णुशर्मा नाम ब्राह्मणः सकल-शास्त्र-पारंगमश् छात्र-सम्̣सदि लब्ध-कीर्तिः | तस्मै समर्पयतु ऎतान् | स नूनम्̣ द्राक् प्रबुढान् करिष्यति इति |

स राजा तद् आकर्ण्य विष्णुशर्माणम् आहूय प्रॊवाच-भॊ भगवन् मद्-अनुग्रहार्थम् ऎतान् अर्थ-शास्त्रम्̣ प्रति द्राग् यथानंय-सद्ड़्शान् विदधासि तथा कुरु | तदाहम्̣ त्वाम्̣ शासन-शतॆन यॊजयिष्यामि |

अथ विष्णु-शर्मा तम्̣ राजानम् आह-दॆव श्रूयताम्̣ मॆ तथ्य-वचनम् | नाहम्̣ विद्या-विक्रयम्̣ शासन-शतॆनापि करॊमि | पुनर् ऎताम्̣स् तव पुत्रान् मास-षट्कॆन यदि नीति-शास्त्रज्ञान् न करॊमि ततः स्व-नाम-त्यागम्̣ करॊमि |

अथासौ राजा ताम्̣ ब्राह्मणस्यासंभाव्याम्̣ प्रतिज्ञाम्̣ श्रुत्वा स-सचिवः प्रह्ड़्ष्टॊ विस्मयांवितस् तस्मै सादरम्̣ तान् कुमारान् समर्प्य पराम्̣ निर्व्ड़्तिम् आजगाम | विष्णुशर्मणापि तान् आदाय तद्-अर्थम्̣ मित्र-भॆद-मित्र-प्राप्ति-काकॊलूकीय-लब्ध-प्रणाश-अपरीक्षित-कारकाणि चॆति पञ्च-तंत्राणि रचयित्वा पाठितास् तॆ राजपुत्राः | तॆपि तांय् अधीत्य मास-षट्कॆन यथॊक्ताः सम्̣व्ड़्त्ताः | ततः प्रभ्ड़्त्य् ऎतत् पञ्चतंत्रकम्̣ नाम नीति-शास्त्रम्̣ बालावबॊधनार्थम्̣ भूतलॆ प्रव्ड़्त्तम् | किम्̣ बहुना-

अधीतॆ य इदम्̣ नित्यम्̣ नीति-शास्त्रम्̣ श्ड़्णॊति च |
न पराभवम् आप्नॊति शक्राद् अपि कदाचन || ।७||

इति कथा-मुखम् |

\end{document}



%%% Local Variables:
%%% mode: latex
%%% TeX-master: t
%%% End:
