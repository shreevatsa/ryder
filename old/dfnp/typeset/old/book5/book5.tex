\documentclass{book}

\begin{document}

ILL-CONSIDERED ACTION
ILL-CONSIDERED ACTION
Here, then, begins Book V, called "Ill-considered
Action." The first verse runs:
Deeds ill-known, ill-recognized,
Ill-accomplished, ill-devised
Thought of these let no man harbor;
Take a warning from the barber.
"How was that?" asked the princes. And Vishnu-
sharman told the following story.
In the southern country is a city called Trumpet-
Flower. In it lived a merchant named Jewel, who
lost his fortune by the decree of fate, though his life
was given to the pursuit of virtue, money, love, and
salvation. The loss of property led to a series of hu-
miliations, so that he sank into utter despondency.
And one night he reflected: "A curse, a curse upon
this state of poverty! For the proverb says:
Conduct, patience, purity,
Manners, loving-kindness, birth,
After money disappears,
Cease to have the slightest worth.
Wisdom, sense, and social charm,
Honest pride and self-esteem,
After money disappears,
All at once become a dream.
427
To the wisdom of the wise
Constant household worries bring
Daily diminution, like
Winter breathed upon by spring.
After money disappears,
Keenest wisdom is at fault,
Choked by daily fuel and clothes,
Oil and butter, rice and salt.
Poor and paltry neighbors scarce
Waken sentiments of scorn,
Like the bubbles on a stream,
Ever dying, ever born.
Yet the rich have license for
All things vulgar and debased:
When the ocean bellows, none
Reprobate his faulty taste."
Having thus set his mind in order, he concluded:
"Under these circumstances, I will abandon life by
self-starvation. What can be made of this calamity
life without money?" With his resolve taken, he went
to sleep.
Now as he slept, a trillion dollars appeared in the
form of a Jain monk, and said: "Good merchant, do
not lose interest. I am a trillion, earned by your
ancestors. Tomorrow morning I will come to your
house in this same form. Then you must club me on
the head, so that I may turn to gold and prove in-
exhaustible."
On awaking in the morning, he spent some time
pondering on his dream: "Let me think. Will this
dream prove true or false? I cannot tell. No doubt
it will prove false, for I think of nothing but money
all day and all night. And the proverb says:
Dreams that do not mean a thing
Come to sick and sorrowing,
Lovelorn, drunk, and worrying."
At this moment a barber arrived to manicure his
wife's nails. And while the barber was busy with his
manicuring, the Jain monk suddenly appeared. When
Jewel perceived the monk, he was delighted and
struck him on the head with a stick of wood that lay
handy. Whereupon the monk turned to gold and im-
mediately fell to the ground.
The merchant then set him up in the middle of the
house, and said to the barber, after handing him a
tip: "My good fellow, you must not tell anybody
what has happened in our house." To this the barber
assented, but when he reached home, he thought:
"Surely, all these naked fellows turn to gold when
clubbed on the head. So tomorrow morning I, too,
will invite a lot of them and club them to death, in
order to get a lot of gold." And the day and the night
dragged away as he meditated his plan.
In the morning he rose and went to a Jain mon-
astery, arranged his upper garment, circumambulated
the Conqueror thrice, sought the ground with his
knees, laid his garment's hem over the gateway of his
mouth, made a profound obeisance, and with an ear-
piercing voice intoned the following hymn:
"The saints victorious endure
Who live by saving knowledge pure,
Who sterilize the mind within
By mind, against the seed of sin.
And further:
The tongue that praiseth Him is blest;
The heart, in Him that seeketh rest;
The hands are blest, and only they,
That e'er to Him due homage pay."
After chanting other hymns also to the same
effect, but in great variety, he sought out the abbot
and dropped on his knees and hands, saying: "Greet-
ings, Your Reverence." From the abbot he received a
benediction for the increase of his virtue, likewise
instructions for a vow that involved the practice of
celibacy. Then he said devoutly: "Holy sir, when
you take your pious walk today, pray come to my
house with your whole company of monks."
"My dear neophyte," replied the abbot, "you
know the holy law. How can you speak so? Do you
take us for Brahmans, that you invite us to eat?
Nay, we wander each day just as it happens, and when
we meet a pious neophyte, enter his house. Begone.
Never speak so again."
"Holy sir," said the barber, "I know it well. I will
do as you say. However, you have many neophytes
engaged in pious labors; while I, for my part, have
made ready strips of canvas adapted to the wrapping
of manuscripts. And for the copying of manuscripts
and the payment of scribes, sufficient money is pro-
vided. In view of this, pray do what seems proper."
And so he started home.
When he arrived there, he got ready cudgels of
acacia wood, placed them in a corner behind the door,
then toward noon he returned to the monastery gate
and waited there. Then as they all came forth in
order of dignity, he besought them as teachers, and
led them to his house. For their part, in their greed
for book-covers and money they passed by their
familiar neophytes, even the pious ones, and joyfully
flocked behind him. Well, there is sense in the verse:
Behold a wonder! Even he
Who lives alone, from kindred free,
With hand for spoon, and air for dress,
Is overcome by greediness.
Then the barber conducted them well into the
house and clubbed them. Under the clubbing some
died, others had their heads broken and began to
bawl. But when the soldiers in the citadel heard the
howling, they said: "Well, well! What is this tre-
mendous hubbub in the middle of town? Come
along!" So they all scampered and saw the monks
rushing from the barber's house, blood streaming over
their bodies. And being asked what it meant, they
told exactly how the barber had behaved.
So the soldiers fettered the barber and carried him
off to court together with such monks as had survived
the slaughter. There the judges questioned him:
"Come, sir! What means this shameful deed by you
committed?" And he replied: "Gentlemen, what else
could I do ?" And with this he related the behavior
of Jewel.
The judges therefore despatched a summonser,
who returned with Jewel. And they questioned him:
"Merchant, why did you kill a certain Jain monk?"
And he in turn gave a full account of the original
monk. Whereupon they said: "Well, well! Let this
villainous barber be impaled. For his act was ill
advised."
When this had been done, they observed:
Deeds ill-known, ill-recognized,
Ill-accomplished, ill-advised
Thought of these let no man harbor;
Take a warning from the barber.
And there is sound sense in this:
Let the well-advised be done;
Ill-advised leave unbegun:
Else, remorse will be let loose,
As with lady and mungoose.
"How was that?" asked Jewel. And they told the
story of
THE LOYAL MUNGOOSE
There was once a Brahman named Godly in a cer-
tain town. His wife mothered a single son and a mun-
goose. And as she loved little ones, she cared for the
mungoose also like a son, giving him milk from her
breast, and salves, and baths, and so on. But she did
not trust him, for she thought: "A mungoose is a
nasty kind of creature. He might hurt my boy." Yes,
there is sense in the proverb:
A son will ever bring delight.
Though bent on folly, passion, spite,
Though shabby, naughty, and a fright.
One day she tucked her son in bed, took a water-
jar, and said to her husband: "Now, Professor, I am
going for water. You must protect the boy from the
mungoose." But when she was gone, the Brahman
went off somewhere himself to beg food, leaving the
house empty.
While he was gone, a black snake issued from his
hole and, as fate would have it, crawled toward the
baby's cradle. But the mungoose, feeling him to be a
natural enemy, and fearing for the life of his baby
brother, fell upon the vicious serpent halfway, joined
battle with him, tore him to bits, and tossed the pieces
far and wide. Then, delighted with his own heroism,
he ran, blood trickling from his mouth, to meet the
mother; for he wished to show what he had done.
But when the mother saw him coming, saw his
bloody mouth and his excitement, she feared that the
villain must have eaten her baby boy, and without
thinking twice, she angrily dropped the water-jar
upon him, which killed him the moment that it
struck. There she left him without a second thought,
and hurried home, where she found the baby safe and
sound, and near the cradle a great black snake, torn
to bits. Then, overwhelmed with sorrow because she
had thoughtlessly killed her benefactor, her son, she
beat her head and breast.
At this moment the Brahman came home with a
dish of rice gruel which he had got from someone in his
begging tour, and saw his wife bitterly lamenting her
son, the mungoose. "Greedy! Greedy!" she cried.
"Because you did not do as I told you, you must now
taste the bitterness of a son's death, the fruit of the
tree of your own wickedness. Yes, this is what hap-
pens to those blinded by greed. For the proverb says:
Indulge in no excessive greed
(A little helps in time of need)
A greedy fellow in the world
Found on his head a wheel that whirled."
"How was that?" asked the Brahman. And his
wife told the story of
THE FOUR TREASURE-SEEKERS
In a certain town in the world were four Brahmans
who lived as the best of friends. And being stricken
with utter poverty, they took counsel together: "A
curse, a curse on this business of being poor! For
The well-served master hates him still;
His loving kinsmen with a will
Abandon him; woes multiply,
While friends and even children fly;
His high-born wife grows cool; the flash
Of virtue dims; brave efforts crash
For him who has no ready cash.
And again:
Charm, courage, eloquence, good looks,
And thorough mastery of books
(If money does not back the same)
Are useless in the social game.
"Better be dead than penniless. As the story goes:
A beggar to the graveyard hied
And there 'Friend corpse, arise,' he cried;
'One moment lift my heavy weight
Of poverty; for I of late
Grow weary, and desire instead
Your comfort: you are good and dead/
The corpse was silent. He was sure
'Twas better to be dead than poor.
"So let us at any cost strive to make money. For
the saying goes:
Money gets you anything,
Gets it in a flash:
Therefore let the prudent get
Cash, cash, cash.
"Now this cash comes to men in six ways. They
are: (i) begging for charity, (2) flunkeyism at a court,
(3) farmwork, (4) the learned professions, (5) usury,
(6) trade.
"However, among all these methods of making
money, trade is the only one without a hitch in it. For
Kings' favor is a thing unstable;
Crows peck at winnings charitable;
You make, in learning the professions.
Too many wearisome concessions
To teachers; farms are too much labor;
In usury you lend your neighbor
The cash which is your life, and therefore
You really live a poor man. Wherefore
I see in trade the only living
That can be truly pleasure-giving.
Hurrah for trade!
"Now profitable trade has seven branches. They
are: (i) false weights and balances, (2) price-boost-
ing* (3) keeping a pawnshop, (4) getting regular cus-
tomers, (5) a stock company, (6) articles de luxe such
as perfumes, (7) foreign trade.
"Now the economists say:
False weights and boosting prices to
An overshameless sum
And constant cheating of one's friends
Are fit for social scum.
And again:
Deposits in the house compel
The pawnshop man to pray:
If you will kill the owner, Lord,
I'll give you what you say.
Likewise:
The holder of a stock reflects
With glee, though one of many:
The wide world's wealth belongs to me;
No other gets a penny.
Furthermore:
Perfumery is first-class ware;
Why deal in gold and such?
Whate'er the cost, you sell it for
A thousand times as much.
"Foreign trade is the affair of the capitalist. As
the book says:
Wild elephants are caught by tame:
So money-kings, devising
A trap for money, capture it
With far-flung advertising.
The brisk commercial traveler,
Who knows the selling game,
.   Invests his money, and returns
With twice or thrice the same.
And again:
The crow, or good-for-naught, or deer,
Afraid of foreign lands,
In heedless slothfulness is sure
To perish where he stands."
Having thus set their minds in order, and resolved
on foreign travel, they said farewell to home and
friends, and started, all four of them. Well, there is
wisdom in the saying:
The man whose mind is money mad.
From all his kinsmen flees;
He hastens from his mother dear;
He breaks his promises;
He even goes to foreign lands
Which he would not elect
And leaves his native country. Well,
What else do you expect ?
So in time they came to the Avanti country, where
they bathed in the waters of the Sipra, and adored the
great god Shiva. As they traveled farther, they met
a master-magician named Terror-Joy. And having
greeted him in proper Brahman fashion, they all ac-
companied him to his monastery cell. There the
magician asked them whence they came, whither they
were going, and what was their object. And they re-
plied: "We are pilgrims, seeking magic power. We
have resolved to go where we shall find enough
money, or death. For the proverb says:
While water is given
By fate out of heaven,
If men dig a well,
It bubbles from hell.
Man's effort (sufficiently great)
Can equal the wonders of fate.
And again:
Success complete
In any feat
Is sure to bless
True manliness.
Man's effort (sufficiently great)
Is just what a dullard calls fate.
There is no toy
Called easy joy,
But man must strain
To body's pain.
Even Vishnu embraces his bride
With arms that the churn-stick has tried.
"So disclose to us some method of getting money,
whether crawling into a hole, or placating a witch, or
living in a graveyard, or selling human flesh, or any-
thing. You are said to have miraculous magic, while
we have boundless daring. You know the saying:
Only the great can aid the great
To win their heart's desire:
Apart from ocean, who could bear
The fierce subaqueous fire?"
So the magician, perceiving their fitness as disci-
ples, made four magic quills, and gave one to each,
saying: "Go to the northern slope of the Himalaya
Mountains. And wherever a quill drops, there the
owner will certainly find a treasure."
Now as they followed his directions, the leader's
quill dropped. And on examining the spot, he found
the soil all copper. So he said: "Look here! Take all
the copper you want." But the others said: "Fool!
What is the good of a thing which, even in quantity,
does not put an end to poverty? Stand up. Let us go
on." And he replied: "You may go. I will accom-
pany you no farther." So he took his copper and was
the first to turn back.
The three others went farther. But they had
traveled only a little way when the leader's quill
dropped. And when he dug down, he found the soil all
silver. At this he was delighted, and cried: "Look!
Take all the silver you want. No need of going
farther." "Fool!" said the other two. "The soil was
copper first, then silver. It will certainly be gold ahead.
This stuff, even in quantity, does not relieve poverty
so much." "You two may go," said he. "I will not
join you." So he took his silver and turned back.
The two went on until one quill dropped. When
the owner dug down, he found the soil all gold. See*
ing this, he was delighted, and said to his companion:
"Look! Take all the gold you want. There is nothing
beyond better than gold." "Fool!" said the other.
"Don't you see the point? First came copper, then
silver, and then gold. Beyond there will certainly be
gems. Stand up. Let us go farther. What is the good
of this stuff? A quantity of it is a mere burden."
"You may go," he replied. "I will stay here and wait
for you."
So the other went on alone. His limbs were scorched
by the rays of the summer sun and his thoughts were
confused by thirst as he wandered to and fro over
the trails in the land of the fairies. At last, on a whirl-
ing platform, he saw a man with blood dripping down
his body; for a wheel was whirling on his head. Then
he made haste and said: "Sir, why do you stand thus
with a wheel whirling on your head? In any case, tell
me if there is water anywhere. I am mad with thirst."
The moment the Brahman said this, the wheel left
the other's head and settled on his own. "My very
dear sir," said he, "what is the meaning of this?" "In
the very same way," replied the other, "it settled on
my head." "But," said the Brahman, "when will it
go away? It hurts terribly."
And the fellow said: "When someone who holds
in his hand a magic quill such as you had, arrives and
speaks as you did, then it will settle on his head."
"Well," said the Brahman, "how long were you
here?" And the other asked: "Who is king in the
world at present?" On hearing the answer, "King
Vinavatsa," he said: "When Rama was king, I was
poverty stricken, procured a magic quill, and came
here, just like you. And I saw another man with a
wheel on his head and put a question to him. The
moment I asked a question (just like you) the wheel
left his head and settled on mine. But I cannot reckon
the centuries."
Then the wheel-bearer asked: "My dear sir, how,
pray, did you get food while standing thus?" "My
dear sir," said the fellow, "the god of wealth, fearful
lest his treasures be stolen, prepared this terror, so
that no magician might come so far. And if any
should succeed in coming, he was to be freed from
hunger and thirst, preserved from decrepitude and
death, and was merely to endure this torture. So now
permit me to say farewell. You have set me free from
a sizable misery. Now I am going home." And he
went.
After he had gone, the gold-finder, wondering why
his companion delayed, eagerly followed his foot-
prints. And having gone but a little way, he saw a
man whose body was drenched with blood, a man
tortured by a cruel wheel whirling on his head---and
this man was his own companion. So he came near
and asked with tears: "My dear fellow, what is the
meaning of this?" "A whim of fate," said the other.
"But tell me," said he, "what has happened." And in
answer to his question, the other told the entire his-
tory of the wheel.
When the friend heard this, he scolded him, say-
ing: "Well, I told you time and again not to do it.
Yet from lack of sense you did not do as I said. In-
deed, there is wisdom in the saying:
Scholarship is less than sense;
Therefore seek intelligence:
Senseless scholars in their pride
Made a lion; then they died."
"How was that?" asked the wheel-bearer. And
the gold-finder told the story of
THE LION-MAKERS
In a certain town were four Brahmans who lived
in friendship. Three of them had reached the far
shore of all scholarship,, but lacked sense. The other
found scholarship distasteful; he had nothing but
sense.
One day they met for consultation. "What is the
use of attainments," said they, "if one does not travel,
win the favor of kings, and acquire money? What-
ever we do, let us all travel."
But when they had gone a little way, the eldest
of them said: "One of us, the fourth, is a dullard, hav-
ing nothing but sense. Now nobody gains the favor-
able attention of kings by simple sense without schol-
arship. Therefore we will not share our earnings with
him. Let him turn back and go home."
Then the second said: "My intelligent friend, you
lack scholarship. Please go home." But the third
said: "No, no. This is no way to behave. For we
have played together since we were little boys. Come
along, my noble friend. You shall have a share of the
money we earn."
With this agreement they continued their journey,
and in a forest they found the bones of a dead lion.
Thereupon one of them said: "A good opportunity to
test the ripeness of our scholarship. Here lies some
kind of creature, dead. Let us bring it to life by means
of the scholarship we have honestly won."
Then the first said: "I know how to assemble the
skeleton." The second said: "I can supply skin,
flesh, and blood." The third said: "I can give it life."
So the first assembled the skeleton, the second
provided skin, flesh, and blood. But while the third
was intent on giving the breath of life, the man of
sense advised against it, remarking: "This is a lion.
If you bring him to life, he will kill every one of
us."
"You simpleton!" said the other, "it is not I who
will reduce scholarship to a nullity." "In that case,"
came the reply, "wait a moment, while I climb this
convenient tree."
When this had been done, the lion was brought to
life, rose up, and killed all three. But the man of
sense, after the lion had gone elsewhere, climbed down
and went home.
"And that is why I say:
Scholarship is less than sense, �...
and the rest of it."
But the wheel-bearer, having heard the story, re-
torted: "Not at all. The reasoning is at fault. For
creatures of very great sense perish if stricken by fate,
while those of very meager intelligence, if protected
by fate, live happily. There is a stanza:
While Hundred-Wit is on a head,
While Thousand-Wit hangs limp and dead,
Your humble Single-Wit, my dear,
Is paddling in the water clear."
"How was that?" asked the gold-finder. And the
wheel-bearer told the story of
HUNDRED-WIT, THOUSAND-WIT, AND
SINGLE-WIT
In a certain pond lived two fishes whose names
were Hundred-Wit and Thousand-Wit. And a frog
named Single-Wit made friends with them. Thus all
three would for some time enjoy at the water's edge
the pleasure of conversation spiced with witticisms,
then would dive into the water again.
One day at sunset they were engaged in conversa-
tion, when fishermen with nets came there, who said
to one another on seeing the pond: "Look! This pond
appears to contain plenty offish, and the water seems
shallow. We will return at dawn/' With this they
went home.
The three friends felt this speech to be dreadful as
the fall of a thunderbolt, and they took counsel to-
gether. The frog spoke first: "Hundred-Wit and
Thousand-Wit, my dear friends, what should we do
now: flee or stick it out?"
At this Thousand-wit laughed and said: "My
good friend, do not be frightened merely because you
have heard words. An actual invasion is not to be
anticipated. Yet should it take place, I will save you
and myself by virtue of my wit. For I know plenty of
tricks in the water." And Hundred-Wit added:
"Yes, Thousand-Wit is quite right. For
Where wind is checked, and light of day,
The wise man's wit soon finds a way.
One cannot, because he has heard a few mere words,
abandon his birthplace, the home of his ancestors.
You must not go away. I will save you by virtue of
my wit."
"Well," said the frog, "I have only a single wit,
and that tells me to flee. My wife and I are going to
some other body of water this very night."
So spoke the frog and under cover of night he went
to another body of water. At dawn the next day came
the fish-catchers, who seemed the servants of Death,
and inclosed the pond with nets. And all the fishes,
turtles, frogs, crabs, and other water-creatures were
caught in the nets and captured. Even Hundred-Wit
and Thousand-Wit fell into a net and were killed,
though they struggled to save their lives by fancy
turns.
On the following day the fishermen gleefully
started home. One of them carried Hundred-Wit,
who was heavy, on his head. Another carried Thou-
sand-Wit tied to a cord. Then the frog, safe in the
throat of a cistern, said to his wife: "Look, darling,
look!
While Hundred-Wit is on a head,
While Thousand-Wit hangs limp and dead,
Your humble Single-Wit, my dear,
Is paddling in the water clear."
"And that is why I say that intelligence is not the
sole determinant of fate."
Then the gold-finder said: "It may be so. Yet a
friend's advice should not be disregarded. But what
happened? Spite of my dissuasion, you would not
stop, such was your greed and pride in your scholar-
ship. Yes, there is sense in the stanza:
Well sung, uncle! Why would you
Not stop when I told you to?
What a necklace! Yes, you wear
Music medals rich and rare."
"How was that?" asked the wheel-bearer. And
the other told the story of
THE MUSICAL DONKEY
In a certain town was a donkey named Prig. In
the daytime he carried laundry packages, but was at
liberty to wander anywhere at night. One night while
wandering in the fields he fell in with a jackal and
made friends. So the two broke through a hedge into
cucumber-beds, and having eaten what they could
hold of that comestible, parted at dawn to go home.
One night the egotistical donkey, standing among
the cucumbers, said to the jackal: "See, nephew!
The night is marvelously fine. I will contribute a
song. What sentiment shall my song express?"
"Don't, uncle," said the jackal. "It might make
trouble, seeing that we are on thieves' business.
Thieves and lovers should keep very quiet. As the
proverb says:
No sleepyhead should pilfer fur,
No invalid, rich provender,
No sneezer should become a thief----
Unless they wish to come to grief.
"Besides, your vocal music is not agreeable, since
it resembles a blast on a conch-shell. The farmers
would hear you from afar, would rise, and would
fetter or kill you. Better keep quiet and eat."
"Come, come!" said the donkey. "Your remarks
prove that you live in the woods and have no musical
taste. Did you never hear this?
Oh, bliss if murmurs sweet to hear
Of music's nectar woo your ear
When darkness flees from moonlight clear
In autumn, and your love is near."
"Very true, uncle," said the jackal. "But your
bray is harsh. Why do a thing that defeats your own
purpose?" "Fool, fool!" answered the donkey. "Do
you think me ignorant of vocal music? Listen to its
systematization, as follows:
Seven notes, three scales, and twenty-one
Are modulations said to be;
Of pitches there are forty-nine,
Three measures, also pauses three;
Caesuras three; and thirty-six
Arrangements of the notes, in fine;
Six apertures; the languages
Are forty; sentiments are nine.
One hundred songs and eighty-five
Are found in songbooks, perfect, pure,
With all accessories complete,
Unblemished in their phrasing sure.
On earth is nothing nobler found,
Nor yet in heaven, than vocal song;
The singing Devil soothes the Lord,
When quivering strings the sound prolong.
"After this, how can you think me lacking in
educated taste? How can you try to hinder me?"
"Very well, uncle," said the jackal. "I will stay by
the gap in the hedge, and look for farmers. You may
sing to heart's content."
When he had done so, the donkey lifted his neck
and began to utter sounds. But the farmers, hearing
the bray of a donkey, angrily clenched their teeth,
snatched cudgels, rushed in, and beat him so that he
fell to the ground. Next they hobbled him by fasten-
ing on his neck a mortar with a convenient hole, then
went to sleep. Presently the donkey stood up, forget-
ting the pain as donkeys naturally do. As the verse
puts it:
With dog, and ass, and horse,
And donkey more than most,
The pain from beatings is
Immediately lost.
Then with the mortar on his neck, he trampled the
hedge and started to run away. At this moment the
jackal, looking on from a safe distance, said with a
smile:
Well sung, uncle! Why would you
Not stop when I told you to?
What a necklace! Yes, you wear
Music medals rich and rare.
"Just so, you would not stop when I advised it."
After listening to this, the wheel-bearer said: "O
my friend, you are quite right. Yes, there is much
wisdom in the verse:
He who, lacking wit, does not
Harken to a friend,
Just like weaver Slow, inclines
To a fatal end."
"How was that?" asked the gold-finder. And the
wheel-bearer told the story of
SLOW, THE WEAVER
In a certain town lived a weaver named Slow. One
day all the pegs in his loom broke. So he took an axe,
and in his search for wood, came to the seashore. There
he found a great sissoo tree, and he thought: "This
seems a good-sized tree. If I cut it down, I can make
plenty of weaving-tools." He therefore lifted his axe
upon it.
Now there was a fairy in the tree who said: "My
friend, this tree is my home. Please spare it. For I
live here in utter happiness, since my body is caressed
by breezes cool from contact with ocean billows."
"But, sir," said the weaver, "what am I to do?
While I lack apparatus made of wood, my family is
pinched by hunger. Therefore, please move else-
where, and quickly. I intend to cut it down."
"Sir," said the fairy, "I have taken a liking to you.
Ask anything you like, but spare this tree."
"In that case," said the weaver, "I will go home
and return after asking my friend and my wife." And
when the fairy consented, the weaver started home.
On entering the town, he encountered his particular
friend, the barber, and said: "My friend, I have won
the favor of a fairy. Tell me what to ask for."
And the barber said: "My dear fellow, if it is
really so, ask for a kingdom. You can be king, and I
will be prime minister. So we shall both taste the de-
lights of this world before those of the world to come."
"Quite so, my friend," replied the weaver. "How-
ever, I shall ask my wife, too." "Don't," said the
barber. "It is a mistake to consult women. As the
saying goes:
Give a woman food and dresses
(Chiefly when her trouble presses);
Give her gems and all things nice;
Do not ask for her advice.
And again:
Where a woman, gambler, child,
As a guide is domiciled,
Death advances, stage by stage
So declares the ancient sage.
And once again:
Only while he does not hear
Woman's whisper in his ear,
May a man a leader be,
Keeping due humility.
Women seek for selfish treasures,
Think of nothing but their pleasures,
Even children by them reckoned
To their selfish comfort second."
And the weaver rejoined: "You may be right.
Still, I shall ask her. She is a good wife."
So he made haste and said to her: "My dear wife,
today we won the favor of a fairy. He offers anything
we want. So I have come to ask you to tell me what
to say to him. Here is my friend, the barber, who tells
me to ask for a kingdom."
"Dear husband/' said she, "what sense have
barbers? Do not take his advice. For the proverb
says:
All advice you may discard
From a barber, child, or bard,
Monk or hermit or musician,
Or a man of base condition.
"Besides, this king-business means a series of
dreadful troubles and involves worry about peace,
war, change of base, entrenchment, alliance, dupli-
city, and other matters. It never gives satisfaction.
And even worse,
His very sons and brothers wish
The slaughter of a king;
As this is kingship's nature, who
Would not reject the thing?0
"Yes," said the weaver, "you are right. But tell
me what to ask for." And she replied: "As it is, you
turn out one piece of cloth a day, and this meets all
our expenses. Now ask for a second pair of arms and
an extra head, so that you may produce one piece of
cloth in front and another behind. The price of one
meets the household expenses, with the price of the
other you may put on style and spend the time in
honor among your peers."
On hearing this, he was delighted and said:
"Splendid, my faithful wife! You have made a splen-
did suggestion. I am determined to follow it."
So the weaver went and laid his request before the
fairy: "Well, sir, if you offer what I wish, pray give
me a second pair of arms and an extra head." And in
the act of speaking he became two-headed and four-
armed.
But as he came home, delight in his heart, the
people thought he was a fiend, and beat him with
clubs and stones and things so that he died.
"And that is why I say:
He who, lacking wit, does not, . .
and the rest of it."
Then the wheel-bearer continued: "Yes, any man
becomes ridiculous when bitten by the demon of
extravagant hope. There is sense in this:
Do not indulge in hopes
Extravagantly high:
Else, whitened like the sire
Of Moon-Lord, you will lie."
"How was that?" asked the gold-finder. And the
other told the story of
THE BRAHMAN'S DREAM
In a certain town lived a Brahman named Seedy,
who got some barley-meal by begging, ate a portion,
and filled a jar with the remainder. This jar he hung
on a peg one night, placed his cot beneath it, and fix-
ing his gaze on the jar, fell into a hypnotic reverie.
"Well, here is a jar full of barley-meal," he
thought. "Now if famine comes, a hundred rupees
will come out of it. With that sum I will get two she-
goats. Every six months they will bear two more she-
goats. After goats, cows. When the cows calve, I will
sell the calves. After cows, buffaloes; after buffaloes,
mares. From the mares I shall get plenty of horses.
The sale of these will mean plenty of gold. The gold
will buy a great house with an inner court. Then
someone will come to my house and offer his lovely
daughter with a dowry. She will bear a son, whom I
shall name Moon-Lord. When he is old enough to
ride on my knee, I will take a book, sit on the stable
roof, and think. Just then Moon-Lord will see me,
will jump from his mother's lap in his eagerness to
ride on my knee, and will go too near the horses.
Then I shall get angry and tell my wife to take the
boy. But she will be busy with her chores and will not
pay attention to what I say. Then I will get up and
kick her."
Being sunk in his hypnotic dream, he let fly such
a kick that he smashed the jar. And the barley-meal
which it contained turned him white all over.
"And that is why I say:
Do not indulge in hopes, ....
and the rest of it."
"Very true, indeed," said the gold-finder. "For
Greedy folk who do not heed
Consequences of a deed,
Suffer disappointment soon;
For example take King Moon."
"How was that?" asked the wheel-bearer. And
the other told the story of
THE UNFORGIVING MONKEY
In a certain city was a king named Moon, who
had a pack of monkeys for his son's amusement.
They were kept in prime condition by daily prov-
ender and pabulum in great variety.
For the amusement of the same prince there was
a herd of rams. One of them had an itching tongue,
so he went into the kitchen at all hours of the day and
night and swallowed everything in sight. And the
cooks would beat him with any stick or other object
within reach.
Now when the chief of the monkeys observed this,
he reflected: "Dear me! This quarrel between ram
and cooks will mean the destruction of the monkeys.
For the ram is a regular guzzler, and when the cooks
are infuriated, they hit him with anything handy.
Suppose some time they find nothing else and beat
him with a firebrand. Then that broad, woolly back
will very easily catch fire. And if the ram, while burn-
ing, plunges into the stable near by, it will blaze
for it is mostly thatch---and the horses will be scorch-
ed. Now the standard work on veterinary science
prescribes monkey-fat to relieve burns on horses.
This being so, we are threatened with death."
Having reached this conclusion, he assembled the
monkeys and said:
"A quarrel of the ram and cooks
Has lately come about;
It threatens every monkey life
Without a shade of doubt.
"Because, if senseless quarrels rend
A house from day to day,
If foes commit an outrage on
A house, and one forgives
Be it from fear or greed---he is
The meanest man that lives.
Now as the elderly monkey wandered about
thirsty, he came to a lake made lovely by clusters of
lotuses. And as he observed it narrowly, he noticed
footprints leading into the lake, but none coming out.
Thereupon he reflected: "There must be some vicious
beast here in the water. So I will stay at a safe dis-
tance and drink through a hollow lotus-stalk."
When he had done so, there issued from the water
a man-eating fiend with a pearl necklace adorning his
neck, who spoke and said: "Sir, I eat everyone who
enters the water. So there is none shrewder than you,
who drink in this fashion. I have taken a liking to
you. Name your heart's desire."
"Sir, " said the monkey, "how many can you eat?"
And the fiend replied: "I can eat hundreds, thou-
sands, myriads, yes, hundreds of thousands, if they
enter the water. Outside, a jackal can overpower
me."
"And I," said the monkey, "I live in mortal
enmity with a king. If you will give me that pearl
necklace, I will awaken his greed with a plausible nar-
rative, and will make that king enter the lake along
with his retinue." So the fiend handed over the pearl
necklace.
Then people saw the monkey roaming over trees
and palace-roofs with a pearl necklace embellishing
his throat, and they asked him: "Well, chief, where
have you spent this long time? Where did you get a
pearl necklace like that? Its dazzling beauty dims the
very sun."
And the monkey answered: "In a spot in the
forest is a shrewdly hidden lake, a creation of the god
of wealth. Through his grace, if anyone bathes there
at sunrise on Sunday, he comes out with a pearl
necklace like this embellishing his throat."
Now the king heard this from somebody, sum-
moned the monkey, and asked: "Is this true, chief?"
"O King," said the monkey, "you have visible proof
in the pearl necklace on my throat. If you, too, could
find a use for one, send somebody with me, and I will
show him."
On hearing this, the king said: "In view of the
facts, I will come myself with my retinue, so that we
may acquire numbers of pearl necklaces." "O King,"
said the monkey, "your idea is delicious."
So the king and his retinue started, greedy for
pearl necklaces. And the king in his palanquin
clasped the monkey to his bosom, showing him
honor as they traveled. For there is wisdom in the
saying:
The educated and the rich,
Befooled by greed,
Plunge into wickedness, then feel
The pinch of need.
And again:
A hundred's mine? A thousand, please.
Thousand? A lakh would give me ease.
A kingdom's power would satisfy
The lakh-lord. Kings would own the sky.
The hair grows old with aging years;
The teeth grow old, the eyes and ears.
But while the aging seasons speed,
One thing is young forever---greed.
At dawn they reached the lake and the monkey
said to the king: "O King, fulfilment comes to those
who enter at sunrise. Let all your attendants be told,
so that they may dash in with one fell swoop. You,
however, must enter with me, for I will pick the
place I found before and show you plenty of pearl
necklaces." So all the attendants entered and were
eaten by the fiend.
Then, as they lingered, the king said to the mon-
key: "Well, chief, why do my attendants linger?"
And the monkey hurriedly climbed a tree before say-
ing to the king: "You villainous king, your attendants
are eaten by a fiend that lives in the water. My
enmity with you, arising from the death of my house-
hold, has been brought to a happy termination. Now
go. I did not make you enter there, because I re-
membered that you were the king. But the proverb
says:
Having suffered an offense,
Give an evil recompense;
For I deem it righteous still,
Evil to repay with ill.
Thus you plotted the death of my household, and I
of yours."
When the king heard this, he hastened home, grief-
stricken. And when the king had gone, the fiend,
fully satisfied, issued from the water, and gleefully
recited a verse:
Very good, my monkey-o!
You won a friend, and killed a foe,
And kept the pearls without a flaw,
By sucking water through a straw.
"And that is why I say:
Greedy folk who do not heed, ....
and the rest of it."
Then the gold-finder continued: "Please bid me
farewell. I wish to go home." But the wheel-bearer
answered: "How can you go, leaving me in this
plight? You know the proverb:
Whoever through hard-heartedness
Deserts a friend in his distress,
For such ingratitude must pay
To hell he treads the certain way."
"That is true," said the gold-finder, "in case one
able to aid deserts a friend in a remediable situation.
But this situation has no human remedy, and I shall
never have the ability to set you free. Besides, the
more I gaze at your face, distorted with pain from the
whirling wheel, the surer I feel that I am going to
leave this spot at once, lest perchance the same ca-
lamity befall me, too. There is some point in this:
To judge by the expression,
Friend monkey, on your face,
You have been caught by Twilight
He lives who wins the race."
"How was that?" asked the wheel-bearer. And
the other told the story of
THE CREDULOUS FIEND
In a certain city lived a king whose name was
Fine-Army. He had a daughter named Pearl, blessed
with the thirty-two marks of perfect beauty.
Now a certain fiend, who wished to carry her off,
came every evening and abused her, but he could not
carry her off because she protected herself by drawing
a magic circle. However, at the hour when he em-
braced her, she experienced trembling, fever, and the
like, the feelings that arise in the presence of a fiend.
While matters were in this state, the fiend once
took his stand in a corner and revealed himself to the
princess, who thereupon said to a girl friend: "Look,
my dear! This is the fiend who comes every evening
at twilight's hour and torments me. Is there any
means of keeping the ruffian at a distance?"
When he heard this, the fiend thought: "Aha! I
am not the only one. There is someone else---and his
name is Twilight---who comes every day to carry her
off. But he cannot do it either. Suppose I take the
form of a horse, go to the stable, and find out what he
looks like and what power he has."
When he had done so, a horse-thief came to the
palace at dead of night. He examined all the horses,
found the fiend-horse the finest, put a bit in his
mouth, and mounted. Meanwhile the fiend was
thinking: "I presume this is the fellow named Twi-
light. He thinks me a vile creature, he is angry, he
has come to kill me. What shall I do?"
While he was thinking, the horse-thief struck him
with a whip. And he was terrified and started to run.
The thief, for his part, after traveling some distance,
tried to stop him by tugging at the bit. And he
thought: "Now if he were a horse, he would mind the
bit. Instead, he goes faster and faster."
When the thief perceived how little he minded the
tugging at the bit, he reflected: "Well, well! Horses
are not like this. This must be a fiend in equine form.
So if I find a spot thick with dust, I will drop. It is
my one chance of life."
While the horse-thief was thinking and praying
to his favorite god, the fiend-horse passed under a
banyan tree. And the thief caught a branch and
stuck. So both of them gained the hope of life
from their separation, and were filled with extreme
delight.
Now in the banyan was a monkey, a friend of the
fiend, who said when he saw the fiend making off:
"Look here! Why do you run from an imaginary
danger? This is your natural food, a man. Eat
him."
On hearing this, the fiend took his own form and
turned about---but his mind was disturbed and his
purpose shaky. And when the thief saw that the
monkey had called him back, he was angry. As the
monkey sat above, and his tail hung down, the thief
took it in his mouth and started to chew very hard.
Then the monkey concluded that he was dealing with
one more powerful than the fiend, and was too fright-
ened to utter a word. In dreadful pain, he could only
shut his eyes tight, clench his teeth, and wait. And
the fiend, observing him in this state, recited the
stanza:
To judge by the expression,
Friend monkey, on your face,
You have been caught by Twilight
He lives who wins the race.
Then the gold-finder continued: "Bid me farewell.
I desire to go home. You may stay here and taste the
fruit of the tree of your waywardness."
"Oh," said the wheel-bearer, "that is uncalled for.
Good or evil comes by fate's decree to men well-be-
haved or wayward. As the old verse puts it:
Blind man, hunchback, and unblest
Princess with an extra breast
Waywardness is prudence, when
Fortune favors wayward men."
"How was that?" asked the gold-finder. And the
wheel-bearer told the story of
THE THREE-BREASTED PRINCESS
In the north country was a city called Honey-
Town, where the king was named Honey-Host. And
once there was born to him a daughter with three
breasts. As soon as he learned of the birth of a three-
breasted girl, he summoned the chamberlain and said:
"Sir, let this girl be exposed in the forest, so that not a
single soul may learn the fact."
To this the chamberlain replied: "O king of kings,
it is a well-known fact that a three-breasted daughter
brings misfortune. In spite of this, the Brahmans
should be summoned and their opinion asked, in order
that no law be offended, whether human or divine.
For the proverb says:
A prudent man should always ask
What is beyond his ken:
A dreadful fiend the Brahman caught,
But let him go again."
"How was that?" asked the king. And the cham-
berlain told the story of
THE FIEND WHO WASHED HIS FEET
In a certain forest lived a fiend named Cruel. One
day he met a Brahman in his wanderings, climbed on
his shoulder, and said: "Now go ahead."
So the terrified Brahman started off with him.
But on observing that the fiend's feet were soft as a
lotus-heart, he asked him: "Sir, why are your feet so
tender?"
And the fiend replied: "I am under a vow never
to touch the ground with my feet until I have washed
them." Soon the Brahman, while meditating a plan
of escape, came to a lake. Here the fiend said: "Sir,
do not stir from this spot until I come forth from the
lake after bathing and worshiping the god."
Thereupon the Brahman thought: "He will be
sure to eat me after his worship. I will hurry away.
For he will not follow me with unwashen feet."
And when he did so, the fiend, not daring to break
his vow, did not follow.
"And that is why I say:
A prudent man should always ask,
and the rest of it."
After listening to this, the king summoned the
Brahmans and said: "Brahmans, a three-breasted
daughter has been born to me. Are any remedial
measures to be taken, or not?" And they replied:
"O King, listen.
A daughter fitted out with limbs
Too numerous or few,
Will lose her character, and will
Destroy her husband, too.
But if the father sees a girl
With triple breast about,
She dooms him to a speedy death
Without a shade of doubt.
"Therefore, O King, shun the sight of her. Give
her to anyone who will marry her, but banish him
from the country. If this is done, there is no offense
to laws human or divine."
When the king had listened to this opinion, he
ordered a proclamation to be made everywhere with
beat of drum, as follows: "Hear ye! There is a three-
breasted princess. To anyone who marries her the
king will give a hundred thousand gold-pieces, but
will exile him." For a long time this proclamation
was made without anyone marrying the princess, who
remained in seclusion and grew to young womanhood.
Now there was a blind man in the city, and as
companion he had a hunchback named Slow, who
guided him with a staff. These two heard the drum
and consulted, saying: "In case we touch that drum,
we get girl and gold. With the gold our life will be
happy. And even if death results from the girl's de-
formity, it will put a final end to the wretchedness of
poverty. For
Until a mortal's belly-pot
Is full, he does not care a jot
For love or music, wit or shame,
For body's care or scholar's name,
For virtue or for social charm,
For lightness or release from harm,
For godlike wisdom, youthful beauty.
For purity or anxious duty."

Not finding a knife, he went up to Slow in the old
way, wrathfully seized him by the feet, whirled him
about his head with every bit of strength he could
muster, and dashed him against the chest of the three-
breasted woman. And the blow from the hunchback's
body forced the third breast in, while the hunchback,
when his hump smashed against her bosom, became
straight.

"And that is why I say:

Blind man, hunchback, . . . .
and the rest of it."

Then the gold-finder said: "Yes, you are quite
right in saying that good fortune always comes
through the favor of fate. Yet, after all, a man
should make fate his ov/n, and not desert prudence,
as you did in rejecting my advice."

With this the gold-finder bade him farewell and
started home.

Here ends Book V, called "Ill-considered Deeds."
The first verse runs:

Deeds ill-known, ill-recognized,
Ill-accomplished, ill-devised---
Thought of these let no man harbor;
Take a warning from the barber.


\end{document}

%%% Local Variables:
%%% mode: latex
%%% TeX-master: t
%%% End:
